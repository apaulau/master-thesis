\newpage

\chapter*{Реферат}
Дипломная работа 42 страницы, 4 главы, 20 рисунков, 5 таблиц, 34 источника, 4 приложения

% Ключевые слова
ВРЕМЕННЫЕ РЯДЫ, ПРОГНОЗИРОВАНИЕ, R, ОПИСАТЕЛЬНЫЕ СТАТИСТИКИ, КОРРЕЛЯЦИОННЫЙ АНАЛИЗ, РЕГРЕССИОННЫЙ АНАЛИЗ, АНАЛИЗ ОСТАТКОВ, ВАРИОГРАММА, КРИГИНГ, КРОСС-ВАЛИДАЦИЯ.

\textit{Объектом исследования} являются наблюдения за температурой воды в озере Баторино в период с 1975 по 2012 гг.

\textit{Цель работы} --- анализ, обработка и прогнозирование с помощью современного языка программирования для статистического анализа --- R.

В процессе работы реализовано веб-приложение, позволяющее решать класс аналогичных поставленной задач. В данном приложении проведён сравнительный анализ современных пакетов статистического анализа, вычислены и проанализированы описательные статистики, произведена подборка закона распределения, проведёны корреляционный и регрессионный анализы, проанализирован ряд остатков, построены и проанализированы различные модели вариограмм и на их основе вычислены прогнозные значения.

Полученные результаты могут быть использованы для дальнейших исследований в различных прикладных областях науки: биологии, химии, гидрологии, --- а также, для анализа экологической ситуации в Нарочанском парке и других регионах.

Реализованное программное обеспечение и предложенные алгоритмы могут использоваться для решения задач со схожими по структуре данных и близких по теме и целям.

Данная работа может быть продолжена для получения модели, более точно описывающей поведение исходного временного ряда. Программное обеспечение и алгоритмы могут быть усовершенствованы в процессе дальнейших исследований, для решения других задач.

\newpage

\chapter*{Рэферат}
ЧАСОВЫЯ ШЭРАГI, ПРАГНАЗАВАННЕ, R, АПІСАЛЬНЫЯ СТАТЫСТЫКI, КАРЭЛЯЦЫЙНЫ АНАЛIЗ, РЭГРЭСIЙЫ АНАЛIЗ, АНАЛIЗ РЭШТКАЎ, ВАРЫЯГРАММА, КРЫГIНГ, КРОСС-ВАЛIДАЦЫЯ

\textit{Аб'ектам даследавання} з'яўляюцца назіранні за тэмпературай вады ў возеры Баторына ў перыяд з 1975 па 2012 гг.

\textit{Мэта працы} --- аналіз, апрацоўка і прагназаванне з дапамогай сучаснай мовы праграмавання для статыстычнага аналізу --- R.

У працэсе працы рэалізаваны вэб-дадатак, якое дазваляе вырашаць клас аналагічных пастаўленай задач. У дадзеным дадатку праведзены параўнальны аналіз сучасных пакетаў статыстычнага аналізу, вылічаны і прааналізаваны апісальныя статыстыкі, праведзена падборка закона размеркавання, праведзен карэляцыйны і рэгрэсійны аналізы, прааналізаваны шэраг рэшткаў, пабудаваны і прааналізаваны розныя мадэлі варыяграмм і на іх аснове вылічаны прагнозныя значэння.

Атрыманыя вынікі могуць быць выкарыстаны для далейшых даследаванняў у розных прыкладных галінах навукі: біялогіі, хіміі, гідралогіі, --- а таксама, для аналізу экалагічнай сітуацыі ў Нарачанскім парку і іншых рэгіёнах.

Рэалізаванае праграмнае забеспячэнне і прапанаваныя алгарытмы могуць выкарыстоўвацца для вырашэння задач з падобнымі па структуры дадзеных і блізкіх па тэме і мэтам.

Дадзеная праца можа быць працягнутая для атрымання мадэлі, больш дакладна апісвае паводзіны зыходнага часовага шэрагу. Праграмнае забеспячэнне і алгарытмы могуць быць удасканалены ў працэсе далейшых даследаванняў, для вырашэння іншых задач.

\newpage

\chapter*{Abstract}
Bachelor's thesis, 42 pages, 4 chapters, 20 figures, 5 tables, 34 sources, 4 appendices.

TIME SERIES, PREDICTION, R, DESCRIPTIONAL STATISTICS, CORRELATIONAL ANALYSIS, REGRESSION ANALYSIS, RESIDUAL ANALYSIS, VARIOGRAMM, KRIGING, CROSS-VALIDATION.

\textit{Object of research} is water temperature observations of Batorino lake in period from 1975 till 2012.

\textit{Research purpose} --- analysis, processing and forecasting with help of modern programming language for statistical analysis --- R.

During the research was implemented rich web-application that allows to solve problems similar to researched within current thesis. With help of implemented application and R programming language were computed and analysed descriptional statistics, was performed destribution analysis and fitting, were conducted correlational and regression analysis, was performed analysis of residual time series, variogram models and based on them prediction values were computed.

Results of this research could be used for further researches in various applied areas of science: biology, chemistry, hydrology, --- and also for analysis of ecology situation at the Narochansky park and other regions.

Implemented web-application and suggested algorythms could be used in case of solving problems with similar by structure data and close by theme and goals of research.

This research could be continued in case of getting model that will be more accurate in describing source time series. Software and algorythms that were obtained during the research could be enhanced during further research for solving different problems.