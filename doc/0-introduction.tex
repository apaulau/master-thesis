%!TEX root = thesis.tex

\newpage

\chapter*{Введение}
\addcontentsline{toc}{chapter}{Введение}

Работа посвящена обработке, исследованию и статистическому анализу реальных временных рядов. В современных условиях выбор этой направленности соответствует необходимости в проведении анализа наблюдений, полученных в течение длительного времени, с математической и, в частности, статистической точки зрения. Часто наличие даже большого количества информации, полученной в процессе каких-либо наблюдений, не всегда позволяет раскрыть те или иные причины и следствия, имеющие место в конкретном случае. Для выявления всех скрытых проблем и свойств объекта, за которым проводилось наблюдение, необходимо провести всесторонний анализ полученной информации. В свою очередь, математический аппарат и его конкретные прикладные части могут позволить не только проанализировать сложившуюся ситуацию, но и постараться дать некоторый прогноз по состоянию объекта в будущем.

В качестве материала для исследования в данной работе используется база данных с реальными наблюдениями, зафиксированными на озёрах, входящих в Нарочанский национальный парк, за период с 1955 по 2012 годы, полученная от учебно-научного центра <<Нарочанская биологическая станция им. Г.Г.Винберга>>. В представленной базе присутствуют данные, полученные в ходе наблюдений за озёрами Баторино, Нарочь и Мястро. Из них для исследования было выбрано озеро Баторино. Данное озеро является уникальным природным объектом, изучение которого позволит решать экологические проблемы не только в региональном, но и глобальном масштабе. Оно располагается у самой границы города Мядель и, вместе с вышеназванными Нарочью и Мястро, а также озерами Белое и Бледное, входит в состав Нарочанской озёрной группы.

В данной работе исследуемым показателем озера Баторино было выбрано значение температуры воды. Температура воды принадлежит к числу наиболее важных и фундаментальных характеристик любого водоёма. Её изменение во времени является одним из главных факторов, отражающих изменения в окружающей среде. Также нужно отметить, что свойства воды непосредственно зависят от температуры, что делает исследование температуры воды еще более актуальной задачей. Данная характеристика оказывает сильное влияние на плотность воды, растворимость в ней газов, минеральных и органических веществ, в том числе кислорода. Растворимость кислорода и насыщенность воды этим газом --- одни из важнейших характеристик для условий обитания в воде живых организмов. В частности, от температуры воды в значительной мере зависит жизнедеятельность рыб: их распределение в водоёме, питание, размножение. К тому же, температура тела рыб, как правило, не превышает температуры окружающей их воды. В то же время, любой водоём как экосистема является средой обитания различных, отличных от рыб, организмов. И поэтому отслеживание всех изменений и влияние этих изменений на их жизнь является крайне важным не только в экологическом смысле, но и в биологическом. Как следствие вышесказанного, изменение температуры с течением времени следует считать одним из важнейших индикаторов изменений, происходящих в экосистеме озера. А исследование данного показателя, в свою очередь, является важнейшим в исследовании различных проблем, возникающих в экосистемах водоёмов. В подтверждение актуальности исследования данной темы можно привести научные работы \cite{Katz2011,OBrien2012a,Subehi2011, ALCANTARA2011, Chokshi2006}, имеющие аналогичное направление.

Среди представленных следует отметить работу \cite{Katz2011}, где в качестве объекта исследования рассматривается крупнейшее в мире озеро --- Байкал, подробно изучается изменение климата в контексте данного озера в период с 1950 по 2012 гг.

В работе \cite{OBrien2012a} исследуется температура воды Великих озёр в Северной Америке, а также исследуется влияние, оказываемое изменением температуры на рыб, обитающих в этих озёрах.

В \cite{Subehi2011} исследуется влияние гидрологических, метеорологических и топологических параметров на изменение температуры воды в озере Цибунту (Индонезия) на основе данных с 2008 по 2009 год.

В работе \cite{ALCANTARA2011} анализируется временной ряд температуры поверхности воды и потоки тепла водоема Итумбиара (Бразилия) в целях улучшения понимания изменений как следствие находящейся там гидроэлектростанции.

В последней упомянутой работе \cite{Chokshi2006} автор исследует на предмет выявления антропогенного влияния на качество воды в крупнейшем озере в Гондурасе --- Йоджоа.

В настоящее время, в условиях глобального потепления и крайне нестабильной климатической ситуации, наблюдения за состоянием озёрных экосистем представляют особую ценность как с научной, так и с практической стороны, поскольку только на основе таких наблюдений возможно выделить последствия антропогенного воздействия на фоне изменения природных факторов. А также получить некоторые заключения по экологической обстановке в определенной области.

Основным инструментом анализа данных в работе является пакет \textbf{R}. Такой выбор был обусловлен тем, что
\begin{itemize}
  \item \textbf{R} является специализированным языком программирования для статистической обработки данных
  \item На сегодняшний день \textbf{R} --- один из самых популярных в статистической среде инструментов анализа данных, имеющий широкую пользовательскую аудиторию, развитую систему поддержки
  \item Пакет постоянно развивается и дополняется современнейшими средствами, моделями и алгоритмами
  \item Бесплатен, свободно распространяется и доступен для всех популярных операционных систем
  \item Обладает развитыми возможностями для работы с графикой
\end{itemize}
Первичный анализ был также выполнен в пакете \textbf{STATISTICA}.