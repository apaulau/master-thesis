% -*- root: thesis.tex -*-

\newpage
\chapter{Случайный процесс и его характеристики}
\label{c:definitions}

\section{Случайный процесс. Стационарность}

Для введения следующих понятий воспользуемся \cite{brillinjer-ts, trush-ts}.

Пусть $ (\Omega, \mathcal{F}, P) $ --- вероятностное пространство, где $\Omega$ является произвольным множеством элементарных событий, $\mathcal{F}$ --- сигма-алгеброй подмножеств $\Omega$, и $P$ --- вероятностной мерой.

\begin{Definition}
\label{def:stochastic-process}
	\textit{Действительным случайным процессом} $ X(t) = X(\omega, t) $ называется семейство действительных случайных величин, заданных на вероятностном пространстве $ (\Omega, \mathcal{F}, P) $, где $ \omega \in \Omega, t \in \mathbb{T}$, где $ \mathbb{T} $ --- некоторое параметрическое множество.

	При $ \omega = \omega_{0} $, $ t \in \mathbb{T} $ $ X(\omega_{0}, t) $ является неслучайной функцией временного аргумента и называется \textit{траекторией случайного процесса}.

	При $ t = t_{0} $, $ \omega \in \Omega $, $ X(\omega, t_{0}) $ является случайной величиной и называется отсчетом случайного процесса.
\end{Definition}

\begin{Definition}
    Если $ \mathbb{T} = \mathbb{R} = (- \infty; + \infty) $, или $ \mathbb{T} \subset \mathbb{R}$, то $ X(t), t \in \mathbb{T} $ называют \textit{случайным процессом с непрерывным временем}.
\end{Definition}

\begin{Definition}
	Если $ \mathbb{T} = \mathbb{Z} = \{ 0, \pm 1, \pm 2, \dots \} $, или $ \mathbb{T} \subset \mathbb{Z} $, то говорят, что $ X(t), t \in \mathbb{T} $, --- \textit{случайный процесс с дискретным временем}.
\end{Definition}

\begin{Definition}
\label{def:distr_func}
	\textit{n-мерной функцией распределения случайного процесса} $ X(t), t \in \mathbb{T} $, называется функция вида
	\begin{equation*}
		F_n(x_1, \dots, x_n; t_1, \dots, t_n) = P \{ X(t_1) < x_1, \dots, X(t_n) < x_n \},
	\end{equation*}
	где $ x_j \in \mathbb{R}, t_j \in \mathbb{T}, j = \overline{1,n} $.
\end{Definition}

\begin{Definition}
	\textit{Математическим ожиданием} случайного процесса $ X(t), t \in \mathbb{T}, $ называется функция вида
	\begin{equation*}
		m(t) = E \{ X(t) \} = \int \limits_{\mathbb{R}} x \, dF_1(x;t), t \in \mathbb{T}.
	\end{equation*}
\end{Definition}

\begin{Definition}
	\textit{Дисперсией} случайного процесса $ X(t), t \in \mathbb{T} $ называется функция вида:
	\begin{equation*}
		V(t) = V \{ X(t) \} = E \{ X(t) - m(t) \}^2 = \int \limits_{\mathbb{R}} (x - m(t))^2 \, dF_1(x; t).
	\end{equation*}
\end{Definition}

\begin{Definition}
	\textit{Ковариационной функцией} случайного процесса $ X(t), t \in \mathbb{T} $ называется функция вида:
	\begin{eqnarray*}
		& cov\{ X(t_1), X(t_2) \} = E \{ (X(t_1) - m(t_1)) (X(t_2) - m(t_2)) \} = \\
		& = \iint \limits_{\mathbb{R}^2} (x_1 - m(t_1)) (x_2 - m(t_2)) \, dF_2(x_1, x_2; t_1, t_2)
	\end{eqnarray*}
\end{Definition}

\begin{Definition}
	\textit{Корреляционной функцией} случайного процесса $ X(t), t \in \mathbb{T} $ называется функция вида:
	\begin{equation*}
		corr\{ X(t_1), X(t_2) \} = E \{ X(t_1)X(t_2) \} = \iint \limits_{\mathbb{R}^2} x_1 x_2 \, dF_2(x_1, x_2; t_1, t_2)
	\end{equation*}
\end{Definition}

\begin{Remark}
\label{rem:corr_cov}
	Имеет место следующее соотношение, связывающее ковариационную и корреляционную функции:
	\begin{equation*}
		corr\{ X(t_1), X(t_2)\} = \frac{cov\{ X(t_1), X(t_2) \}}{\sqrt{ V\{ X(t_1) \} V\{ X(t_2) \} }},
	\end{equation*}
	где $ X(t), t \in \mathbb{T} $, --- случайный процесс.
\end{Remark}

\begin{Definition}
	Случайный процесс $ X (t), t \in \mathbb{T} $, называется \textit{стационарным в широком смысле}, если $ \exists E \{ x^2(t) \} < \infty, t \in \mathbb{T} $, и
	\begin{enumerate}
		\item $ m(t) = E \{ x(t) \} = m = const, t \in \mathbb{T} $;
		\item $ cov(t_1, t_2) = cov(t_1 - t_2), t_1,t_2 \in \mathbb{T} $.
	\end{enumerate}
\end{Definition}

\begin{Definition}
	Случайный процесс $ X(t), t \in \mathbb{T} $, называется \textit{стационарным в узком смысле}, если $ \forall n \in \mathbb{N} $, $ \forall t_1, \dots, t_n \in \mathbb{T} $, $ \forall \tau, t_1 + \tau, \dots, t_n + \tau \in \mathbb{T} $ выполняется соотношение:
	\begin{equation*}
		F_n(x_1, \dots, x_n; t_1, \dots, t_n) = F_n(x_1, \dots, x_n; t_1 + \tau , \dots, t_n + \tau).
	\end{equation*}
\end{Definition}

\begin{Remark}
	Если случайный процесс $ X(t), t \in \mathbb{T} $, является стационарным в узком смысле и $ \exists E \{ x^2(t) \} < \infty, t \in \mathbb{T} $, то он будет стационарным и в широком смысле, но не наоборот.
\end{Remark}

\section{Вариограмма и внутренне стационарный случайный процесс}
\label{sec:variogramAndInnerStationarity}

\begin{Definition}
    \textit{Вариограммой} случайного процесса $ X(t), t \in \mathbb{T} $ называется функция вида
	\begin{equation}
	    2 \gamma (h) = V \{ X(t + h) - X(t) \}, t, h \in \mathbb{T}.
	\end{equation}

	При этом функция $ \gamma (h), h \in \mathbb{T} $, называется \textit{семивариограммой}.
\end{Definition}

\begin{Definition}
	Случайный процесс $ X(t), t \in \mathbb{T} $, называется \textit{внутренне стационарным}, если справедливы следующие равенства:
	\begin{equation}
		E \{ X(t_1) - X(t_2) \} = 0,
	\end{equation}
	\begin{equation}
	    V \{ X(t_1) - X(t_2) \} = 2 \gamma (t_1 - t_2),
	\end{equation}
	где $ 2 \gamma(t_1 - t_2) $ --- вариограмма рассматриваемого процесса, $ t_1, t_2 \in \mathbb{T} $.
\end{Definition}

\begin{Definition}
    Случайный процесс $ X(t), t \in \mathbb{T} $ называется гауссовским, если любые $ n $ его отсчетов $ X(t_1), X(t_2), \ldots, X(t_n), $ где $ t_1, t_2, \ldots, t_n \in \mathbb{T} $ имеют n-мерное нормальное распределение, то есть
	\begin{equation}
	    F_n(\cdot) \equiv \Phi_n(\cdot) \forall n.
	\end{equation}
\end{Definition}

%TODO в замечании вставить ссылку на какую-нибудь из статей Цеховой, чтобы было понятно, откуда оно взялось
\begin{Remark}
	Если $ X(t), t \in \mathbb{T} $, --- внутренне стационарный гауссовский случайный процесс, то
	\begin{equation*}
		( X(t + h) - X(t) )^2 = 2 \gamma(h)\chi_1^2,
	\end{equation*}
	где $\chi_1^2$ --- случайная величина, распределенная по закону \textit{хи-квадрат} с одной степенью свободы.
\end{Remark}

При этом
\begin{equation}
\label{eq:E_diff_inc}
	E \{ X(t + h) - X(t) \}^2 = 2 \gamma(h),
\end{equation}
\begin{equation}
\label{eq:V_diff_inc}
	V \{ X(t + h) - X(t) \}^2 = 2 (2 \gamma(h))^2.
\end{equation}

В дальнейшем в данной работе будем рассматривать случайные процессы с дискретным временем.