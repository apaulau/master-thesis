% -*- root: thesis.tex -*-

\newpage

\chapter{Оценка вариограммы внутренне стационарного гауссовского случайного процесса и ее свойства}
\label{c:variogram_estimation}

Рассмотрим внутренне стационарный гауссовский случайный процесс с дискретным временем $ X(t), t \in \mathbb{Z} $.

Вариограмма процесса $ X(t) , 2 \gamma(h) $, является неизвестной и, не нарушая общности, далее считаем $ m(t) \equiv 0 $, $ V(t) \equiv \sigma^2 , t \in \mathbb{Z}$.

Наблюдается процесс $ X(t), t \in \mathbb{Z} $, и регистрируются наблюдения $ X(1), \dots, X(n) $ в последовательные моменты времени $ 1, \dots, n $.

В качестве оценки вариограммы рассмотрим статистику, предложенную Матероном \cite{matheron1980}:
\begin{equation}
\label{eq:var_estimation}
	2 \tilde{\gamma}(h) = \frac{1}{n - h} \sum_{t = 1}^{n - h}(X(t + h) - X(t))^2, \quad h = \overline{0, n - 1},
\end{equation}
где $ \tilde{\gamma}(-h) = \tilde{\gamma}(h), h = \overline{0, n - 1}$; $ \tilde{\gamma}(h) = 0, \vert h \vert \ge n $.

\section{Первые два момента оценки вариограммы} % (fold)
\label{sec:variogram_moments}

Найдем выражения для первых двух моментов оценки вариограммы \eqref{eq:var_estimation}.
\begin{Theorem}
	Для оценки $ 2 \tilde{\gamma}(h) $, представленной равенством \eqref{eq:var_estimation}, имеют место следующие соотношения:
	\begin{equation}
	\label{eq:est_ex}
		E \{2 \tilde{\gamma}(h) \} = 2 \gamma(h), % DIRTY HACK
	\end{equation}
	\begin{equation*}
		cov(2 \tilde{\gamma}(h_1), 2 \tilde{\gamma}(h_2)) =
	\end{equation*}
	\begin{equation}
	\label{eq:est_cov}
		= \frac{2}{(n - h_1)(n - h_2)} \sum_{t = 1}^{n - h_1}\sum_{s = 1}^{n - h_2} (\gamma(t - h_2 - s) + \gamma(t + h_1 - s) - \gamma(t - s) - \gamma(t + h_1 - s - h_2))^2,
	\end{equation}
	\begin{equation}
	\label{eq:est_var}
		V \{ 2 \tilde{\gamma}(h) \} = \frac{2}{(n-h)^2}\sum_{t,s = 1}^{n - h} ( \gamma(t - h - s) + \gamma(t + h - s) - 2\gamma(t - s) )^2,
	\end{equation}
	где $ \gamma(h), h \in \mathbb{R} $, --- семивариограмма процесса $ X(t), t \in \mathbb{R}$, $ h, h_1, h_2 = \overline{0, n - 1} $.
\end{Theorem}
\begin{proof}

Вычислим первый момент введённой оценки \eqref{eq:var_estimation}, используя свойства математического ожидания:
\begin{equation*}
	E \{ 2 \tilde{\gamma}(h) \} = E \{ \frac{1}{n - h} \sum_{t = 1}^{n - h}(X(t + h) - X(t))^2) \} = \frac{1}{n - h} \sum_{t = 1}^{n - h} E \{ (X(t + h) - X(t))^2 \}.
\end{equation*}

Из равенства \eqref{eq:E_diff_inc} получаем, что
\begin{equation*}
	E \{ 2 \tilde{\gamma}(h) \} = \frac{1}{n - h} \sum_{t = 1}^{n - h} 2 \gamma(h) = 2 \gamma(h).
\end{equation*}

Таким образом, оценка \eqref{eq:var_estimation} является \textbf{несмещённой} для вариограммы рассматриваемого процеса.

Найдём второй момент оценок вариограммы при различных значениях $h$:
\begin{eqnarray}
\label{eq:cov_support}
\nonumber
	cov\{ 2 \tilde{\gamma}(h_1), 2 \tilde{\gamma}(h_2) \} = E\{ (2 \tilde{\gamma}(h_1) - E\{ 2 \tilde{\gamma}(h_1) \}) (2 \tilde{\gamma}(h_2) - E\{ 2 \tilde{\gamma}(h_2) \}) \} = \\
\nonumber
	= E\{ \frac{1}{n - h_1} \sum_{t = 1}^{n - h_1}((X(t + h_1) - X(t))^2 - E\{ (X(t + h_1) - X(t))^2 \}) \times \\
\nonumber
	\times \frac{1}{n - h_2} \sum_{s = 1}^{n - h_2}((X(s + h_2) - X(s))^2 - E\{ (X(s + h_2) - X(s))^2 \}) \} = \\
	= \frac{1}{(n - h_1)(n - h_2)} \sum_{t = 1}^{n - h_1}\sum_{s = 1}^{n - h_2} cov\{ (X(t + h_1) - X(t))^2, (X(s + h_2) - X(s))^2 \}
\end{eqnarray}

Из свойства \ref{rem:corr_cov} корреляции получаем, что
\begin{multline*}
	cov\{ (X(t + h_1) - X(t))^2, (X(s + h_2) - X(s))^2 \} = corr\{(X(t + h_1) - X(t))^2, (X(s + h_2) - X(s))^2 \} \times \\
	\times \sqrt{V\{ (X( t + h_1) - X(t))^2 \} V\{ (X(s + h_2) - X(s))^2 \}}
\end{multline*}

Принимая во внимание \eqref{eq:V_diff_inc} и предыдущее соотношение, из \eqref{eq:cov_support} получаем:
\begin{multline*}
	cov\{ (X(t + h_1) - X(t))^2, (X(s + h_2) - X(s))^2 \} = \frac{2 (2\gamma(h_1))(2\gamma(h_2))}{(n - h_1)(n - h_2)} \times \\
	\times \sum_{t = 1}^{n - h_1}\sum_{s = 1}^{n - h_2} corr\{(X(t + h_1) - X(t))^2, (X(s + h_2) - X(s))^2 \}
\end{multline*}

Далее воспользуемся леммой 1 из \cite{tsekhavaya-brest}:
\begin{multline*}
	cov\{ 2 \tilde{\gamma}(h_1), 2 \tilde{\gamma}(h_2) \} = \\
	= \frac{2 (2\gamma(h_1))(2\gamma(h_2))}{(n - h_1)(n - h_2)} \sum_{t = 1}^{n - h_1}\sum_{s = 1}^{n - h_2} (corr\{(X(t + h_1) - X(t))^2, (X(s + h_2) - X(s))^2 \})^2 = \\
	= \frac{2 (2\gamma(h_1))(2\gamma(h_2))}{(n - h_1)(n - h_2)}\sum_{t = 1}^{n - h_1}\sum_{s = 1}^{n - h_2} ( \frac{cov\{ X(t + h_1) - X(t), X(s + h_2) - X(s) \}}{\sqrt{V\{ X( t + h_1) - X(t) \} V\{ X(s + h_2) - X(s) \}}} )^2
\end{multline*}

Воспользовавшись леммой 3 из \cite{tsekhavaya-brest} и определением корреляционной функции, получаем соотношение \eqref{eq:est_cov}:
\begin{eqnarray}
\nonumber
\label{eq:cov_base}
	cov\{ 2 \tilde{\gamma}(h_1), 2 \tilde{\gamma}(h_2) \} = \\
	= \frac{2}{(n - h_1)(n - h_2)} \sum_{t = 1}^{n - h_1}\sum_{s = 1}^{n - h_2} (\gamma(t - h_2 - s) + \gamma(t + h_1 - s) - \gamma(t - s) - \gamma(t + h_1 - s - h_2))^2
\end{eqnarray}

Отсюда нетрудно получить соотношение \eqref{eq:est_var} для дисперсии оценки вариограммы $ 2 \tilde{\gamma}(h) $, если положить $ h_1 = h_2 = h $:
\begin{multline*}
\nonumber
	V \{ 2 \tilde{\gamma}(h) \} = \frac{2}{(n - h)^2}\sum_{t,s = 1}^{n - h} ( \gamma(t - h - s) + \gamma(t + h - s) - \gamma(t - s) - \gamma(t + h - s - h) )^2 = \\
	= \frac{2}{(n-h)^2}\sum_{t,s = 1}^{n - h} ( \gamma(t - h - s) + \gamma(t + h - s) - 2\gamma(t - s) )^2.
\end{multline*}

\end{proof}
% section variogram_moments (end)

\section{Асимптотическое поведение оценки вариограммы} % (fold)
\label{sec:new_section}

Проанализируем асимптотическое поведение моментов второго порядка оценки \eqref{eq:var_estimation}.

\begin{Theorem}
	Если имеет место соотношение
	\begin{equation}
	\label{eq:var_abs}
		\sum_{m = -\infty}^{+\infty} \vert \gamma(h) \vert < \infty,
	\end{equation}
	то
	\begin{equation*}
		\lim_{n \to \infty} (n - \min\{ h_1, h_2 \}) cov\{ 2 \tilde{\gamma}(h_1), 2 \tilde{\gamma}(h_2) \} = % DIRTY HACK
	\end{equation*}
	\begin{equation}
	\label{eq:asymptotic_cov}
		= 2 \sum_{m = -\infty}^{+\infty} \gamma(m - h_2) + \gamma(m + h_1) - \gamma(m) - \gamma(m + h_1 - h_2))^2,
	\end{equation}
	\begin{equation}
	\label{eq:asymptotic_var}
		\lim_{n \to \infty} (n - h) V\{ 2 \tilde{\gamma}(h) \} = 2 \sum_{m = -\infty}^{+\infty} \gamma(m - h) + \gamma(m + h) - 2 \gamma(m))^2.
	\end{equation}
	где $ \gamma(h), h \in \mathbb{R} $, --- семивариограмма процесса $ X(t), t \in \mathbb{R}$, $ h, h_1, h_2 = \overline{0, n - 1} $.
\end{Theorem}
\begin{proof}

В \eqref{eq:cov_base} сделаем следующую замену переменных
\begin{equation*}
	t = t, \quad m = t - s.
\end{equation*}
Получим
\begin{multline}
\label{eq:cov_split}
	cov\{ 2 \tilde{\gamma}(h_1), 2 \tilde{\gamma}(h_2) \} = \\
	= \frac{2}{(n - h_1) (n - h_2)} \sum_{t = 1}^{n - h_1}\sum_{t - m = 1}^{n - h_2} (\gamma(m - h_2) + \gamma(m + h_1) - \gamma(m) - \gamma(m + h_1 - h_2))^2
\end{multline}

Таким образом, в зависимости от $h_1$ и $h_2$, возможны два случая: $h_1 > h_2$ и $h_1 < h_2$.

\begin{figure}[H]
	\centering
	\begin{tikzpicture}[baseline, font=\tiny]
		\begin{axis}[
		    title = $h_1 > h_2$,
		    xlabel = {$t$},
		    ylabel = {$m$},
		    xmin=-1.5, xmax=9.5,
		    ymin=-5, ymax=3,
			axis x line=middle,
			axis y line=middle,
		    xticklabels = {0, , , , , $n - h_1$},
		    yticklabels = { ,$1 - n + h_2$, $h_2 - h_1$, 0, $n - h_1 -1$},
		    x tick label style={anchor=north west},
		]
			\addplot coordinates {
				(1, 0)
				(8, 2)
				(8, 0)
				(8, -2)
				(1, -4)
				(1, 0)
			};
			\addplot[dashed, draw=gray, domain=0:8]{-2};
			\addplot[dashed, draw=gray, domain=0:8]{2};
			\node at (axis cs:3.5, 1.3) [rotate=17, anchor=north east] {$m = t - 1$};
			\node at (axis cs:4.5, -2.4) [rotate=19, anchor=north east] {$m = t - n + h_2$};
		\end{axis}
		\end{tikzpicture}
		\hspace{0.15cm}
		\begin{tikzpicture}[baseline, font=\tiny]
		\begin{axis}[
		    title = $h_1 < h_2$,
		    xlabel = {$t$},
		    ylabel = {$m$},
		    xmin=-1.5, xmax=9.5,
		    ymin=-3, ymax=5,
			axis x line=middle,
			axis y line=middle,
		    xticklabels = {0, , , , , $n - h_1$},
		    yticklabels = { ,$1 - n + h_2$, 0, $h_2 - h_1$, $n - h_1 - 1$}
		]
			\addplot coordinates {
				(1, 0)
				(8, 4)
				(8, 2)
				(1, -2)
				(1, 0)
			};
			\addplot[dashed, draw=gray, domain=0:8]{2};
			\addplot[dashed, draw=gray, domain=0:8]{4};
			\node at (axis cs:3.5, 2) [rotate=30, anchor=north east] {$m = t - 1$};
			\node at (axis cs:3.7, 0.2) [rotate=32, anchor=north east] {$m = t - n + h_2$};
		\end{axis}
	\end{tikzpicture}

	\caption{Замена переменных}
	\label{fig:label}
\end{figure}

Рассмотрим первый случай: $h_1 > h_2$. Поменяем порядок суммирования в \eqref{eq:cov_split}.
\begin{multline*}
	cov\{ 2 \tilde{\gamma}(h_1), 2 \tilde{\gamma}(h_2) \} = \frac{2}{(n - h_1) (n - h_2)} \times \\
	\times (\sum_{m = 1 - n + h_2}^{h_2 - h_1 - 1}\sum_{t = 1}^{m + n - h_2}(\gamma(m - h_2) + \gamma(m + h_1) - \gamma(m) - \gamma(m + h_1 - h_2))^2 + \\
	+ \sum_{m = h_2 - h_1}^{0}\sum_{t = 1}^{n - h_1}(\gamma(m - h_2) + \gamma(m + h_1) - \gamma(m) - \gamma(m + h_1 - h_2))^2 + \\
	+ \sum_{m = 1}^{n - h_1 - 1}\sum_{t = m + 1}^{n - h_1}(\gamma(m - h_2) + \gamma(m + h_1) - \gamma(m) - \gamma(m + h_1 - h_2))^2)
\end{multline*}

Заметим, что выражение под знаком суммы не зависит от $t$, получим:
\begin{multline*}
	cov\{ 2 \tilde{\gamma}(h_1), 2 \tilde{\gamma}(h_2) \} = \frac{2}{(n - h_1) (n - h_2)} \times \\
	\times (\sum_{m = 1 - n + h_2}^{h_2 - h_1 - 1}(m + n - h_2)(\gamma(m - h_2) + \gamma(m + h_1) - \gamma(m) - \gamma(m + h_1 - h_2))^2 + \\
	+ (n - h_1)\sum_{m = h_2 - h_1}^{0}(\gamma(m - h_2) + \gamma(m + h_1) - \gamma(m) - \gamma(m + h_1 - h_2))^2 + \\
	+ \sum_{m = 1}^{n - h_1 - 1}(n - h_1 - m)(\gamma(m - h_2) + \gamma(m + h_1) - \gamma(m) - \gamma(m + h_1 - h_2))^2)
\end{multline*}

Вынесем $ n - h_1 $ из каждого слагаемого:
\begin{multline*}
	cov\{ 2 \tilde{\gamma}(h_1), 2 \tilde{\gamma}(h_2) \} = \frac{2}{n - h_2} \times \\
	\times (\sum_{m = 1 - n + h_2}^{h_2 - h_1 - 1} (1 + \frac{h_1 + m - h_2}{n - h_1})(\gamma(m - h_2) + \gamma(m + h_1) - \gamma(m) - \gamma(m + h_1 - h_2))^2 + \\
	+ \sum_{m = h_2 - h_1}^{0}(\gamma(m - h_2) + \gamma(m + h_1) - \gamma(m) - \gamma(m + h_1 - h_2))^2 + \\
	+ \sum_{m = 1}^{n - h_1 - 1}(1 - \frac{m}{n - h_1})(\gamma(m - h_2) + \gamma(m + h_1) - \gamma(m) - \gamma(m + h_1 - h_2))^2)
\end{multline*}

Раскроем скобки под знаками сумм:
\begin{multline*}
	cov\{ 2 \tilde{\gamma}(h_1), 2 \tilde{\gamma}(h_2) \} = \frac{2}{n - h_2} (\sum_{m = 1 - n + h_2}^{h_2 - h_1 - 1} (\gamma(m - h_2) + \gamma(m + h_1) - \gamma(m) - \gamma(m + h_1 - h_2))^2 + \\
	+ \frac{1}{n - h_1} \sum_{m = 1 - n + h_2}^{h_2 - h_1 - 1} (h_1 + m - h_2)(\gamma(m - h_2) + \gamma(m + h_1) - \gamma(m) - \gamma(m + h_1 - h_2))^2 + \\
	+ \sum_{m = h_2 - h_1}^{0}(\gamma(m - h_2) + \gamma(m + h_1) - \gamma(m) - \gamma(m + h_1 - h_2))^2 + \\
	+ \sum_{m = 1}^{n - h_1 - 1}(\gamma(m - h_2) + \gamma(m + h_1) - \gamma(m) - \gamma(m + h_1 - h_2))^2 - \\
	- \frac{1}{n - h_1} \sum_{m = 1}^{n - h_1 - 1} m (\gamma(m - h_2) + \gamma(m + h_1) - \gamma(m) - \gamma(m + h_1 - h_2))^2)
\end{multline*}

Приведём подобные:
\begin{multline*}
	cov\{ 2 \tilde{\gamma}(h_1), 2 \tilde{\gamma}(h_2) \} = \frac{2}{n - h_2} (\sum_{m = 1 - n + h_2}^{n - h_1 - 1} (\gamma(m - h_2) + \gamma(m + h_1) - \gamma(m) - \gamma(m + h_1 - h_2))^2 + \\
	+ \frac{1}{n - h_1} \sum_{m = 1 - n + h_2}^{h_2 - h_1 - 1} (h_1 + m - h_2)(\gamma(m - h_2) + \gamma(m + h_1) - \gamma(m) - \gamma(m + h_1 - h_2))^2 - \\
	- \frac{1}{n - h_1} \sum_{m = 1}^{n - h_1 - 1} m (\gamma(m - h_2) + \gamma(m + h_1) - \gamma(m) - \gamma(m + h_1 - h_2))^2)
\end{multline*}

Во втором слагаемом сделаем замену переменных $ m = -(m + h_1 - h_2) $, получим:
\begin{multline*}
	cov\{ 2 \tilde{\gamma}(h_1), 2 \tilde{\gamma}(h_2) \} = \frac{2}{n - h_2} (\sum_{m = 1 - n + h_2}^{n - h_1 - 1} (\gamma(m - h_2) + \gamma(m + h_1) - \gamma(m) - \gamma(m + h_1 - h_2))^2 - \\
	- \frac{1}{n - h_1} \sum_{m = 1}^{n - h_1 - 1} m (\gamma(-m - h_1) + \gamma(-m + h_2) - \gamma(-m - h_1 + h_2) - \gamma(-m))^2 - \\
	- \frac{1}{n - h_1} \sum_{m = 1}^{n - h_1 - 1} m (\gamma(m - h_2) + \gamma(m + h_1) - \gamma(m) - \gamma(m + h_1 - h_2))^2)
\end{multline*}

По определению семивариограммы, $ \gamma(-h) = \gamma(h) $, тогда
\begin{multline*}
	cov\{ 2 \tilde{\gamma}(h_1), 2 \tilde{\gamma}(h_2) \} = \frac{2}{n - h_2} (\sum_{m = 1 - n + h_2}^{n - h_1 - 1} (\gamma(m - h_2) + \gamma(m + h_1) - \gamma(m) - \gamma(m + h_1 - h_2))^2 - \\
	- \frac{2}{n - h_1} \sum_{m = 1}^{n - h_1 - 1} m (\gamma(m - h_2) + \gamma(m + h_1) - \gamma(m) - \gamma(m + h_1 - h_2))^2)
\end{multline*}

Аналогично для случая $h_1 < h_2$:
\begin{multline*}
	cov\{ 2 \tilde{\gamma}(h_1), 2 \tilde{\gamma}(h_2) \} = \frac{2}{(n - h_1) (n - h_2)} \times \\
	\times (\sum_{m = 1 - n + h_2}^{0}\sum_{t = 1}^{m + n - h_2}(\gamma(m - h_2) + \gamma(m + h_1) - \gamma(m) - \gamma(m + h_1 - h_2))^2 + \\
	+ \sum_{m = 1}^{h_2 - h_1}\sum_{t = m + 1}^{m + n - h_2}(\gamma(m - h_2) + \gamma(m + h_1) - \gamma(m) - \gamma(m + h_1 - h_2))^2 + \\
	+ \sum_{m = h_2 - h_1 + 1}^{n - h_1 - 1}\sum_{t = m + 1}^{n - h_1}(\gamma(m - h_2) + \gamma(m + h_1) - \gamma(m) - \gamma(m + h_1 - h_2))^2)
\end{multline*}

Выражение под знаком суммы не зависит от $t$:
\begin{multline*}
	cov\{ 2 \tilde{\gamma}(h_1), 2 \tilde{\gamma}(h_2) \} = \frac{2}{(n - h_1) (n - h_2)} \times \\
	\times (\sum_{m = 1 - n + h_2}^{0}(m + n - h_2)(\gamma(m - h_2) + \gamma(m + h_1) - \gamma(m) - \gamma(m + h_1 - h_2))^2 + \\
	+ (n - h_2)\sum_{m = 1}^{h_2 - h_1}(\gamma(m - h_2) + \gamma(m + h_1) - \gamma(m) - \gamma(m + h_1 - h_2))^2 + \\
	+ \sum_{m = h_2 - h_1 + 1}^{n - h_1 - 1}(n - h_1 - m)(\gamma(m - h_2) + \gamma(m + h_1) - \gamma(m) - \gamma(m + h_1 - h_2))^2)
\end{multline*}

Вынесем $ n - h_2 $ из каждого слагаемого:
\begin{multline*}
	cov\{ 2 \tilde{\gamma}(h_1), 2 \tilde{\gamma}(h_2) \} = \frac{2}{n - h_1} \times \\
	\times (\sum_{m = 1 - n + h_2}^{0} (1 + \frac{m}{n - h_2})(\gamma(m - h_2) + \gamma(m + h_1) - \gamma(m) - \gamma(m + h_1 - h_2))^2 + \\
	+ \sum_{m = 1}^{h_2 - h_1}(\gamma(m - h_2) + \gamma(m + h_1) - \gamma(m) - \gamma(m + h_1 - h_2))^2 + \\
	+ \sum_{m = h_2 - h_1 + 1}^{n - h_1 - 1}(1 + \frac{h_2 - h_1 - m}{n - h_2})(\gamma(m - h_2) + \gamma(m + h_1) - \gamma(m) - \gamma(m + h_1 - h_2))^2)
\end{multline*}

Раскроем скобки под знаками сумм:
\begin{multline*}
	cov\{ 2 \tilde{\gamma}(h_1), 2 \tilde{\gamma}(h_2) \} = \frac{2}{n - h_1} (\sum_{m = 1 - n + h_2}^{0} (\gamma(m - h_2) + \gamma(m + h_1) - \gamma(m) - \gamma(m + h_1 - h_2))^2 + \\
	+ \frac{1}{n - h_2} \sum_{m = 1 - n + h_2}^{0} (m)(\gamma(m - h_2) + \gamma(m + h_1) - \gamma(m) - \gamma(m + h_1 - h_2))^2 + \\
	+ \sum_{m = 1}^{h_2 - h_1}(\gamma(m - h_2) + \gamma(m + h_1) - \gamma(m) - \gamma(m + h_1 - h_2))^2 + \\
	+ \sum_{m = h_2 - h_1 + 1}^{n - h_1 - 1}(\gamma(m - h_2) + \gamma(m + h_1) - \gamma(m) - \gamma(m + h_1 - h_2))^2 + \\
	+ \frac{1}{n - h_2} \sum_{m = h_2 - h_1 + 1}^{n - h_1 - 1} (h_2 - h_1 -m) (\gamma(m - h_2) + \gamma(m + h_1) - \gamma(m) - \gamma(m + h_1 - h_2))^2)
\end{multline*}

Приведём подобные:
\begin{multline*}
	cov\{ 2 \tilde{\gamma}(h_1), 2 \tilde{\gamma}(h_2) \} = \frac{2}{n - h_1} (\sum_{m = 1 - n + h_2}^{n - h_1 - 1} (\gamma(m - h_2) + \gamma(m + h_1) - \gamma(m) - \gamma(m + h_1 - h_2))^2 + \\
	+ \frac{1}{n - h_2} \sum_{m = 1 - n + h_2}^{0} m (\gamma(m - h_2) + \gamma(m + h_1) - \gamma(m) - \gamma(m + h_1 - h_2))^2 + \\
	+ \frac{1}{n - h_2} \sum_{m = h_2 - h_1 + 1}^{n - h_1 - 1} (h_2 - h_1 - m) (\gamma(m - h_2) + \gamma(m + h_1) - \gamma(m) - \gamma(m + h_1 - h_2))^2)
\end{multline*}

Во втором слагаемом сделаем замену переменных $ m = -m $, в третьем $ m = m - h_1 + h_2 $, получим:
\begin{multline*}
	cov\{ 2 \tilde{\gamma}(h_1), 2 \tilde{\gamma}(h_2) \} = \frac{2}{n - h_1} (\sum_{m = 1 - n + h_2}^{n - h_1 - 1} (\gamma(m - h_2) + \gamma(m + h_1) - \gamma(m) - \gamma(m + h_1 - h_2))^2 - \\
	- \frac{1}{n - h_2} \sum_{m = 0}^{n - h_2 - 1} m (\gamma(-m - h_2) + \gamma(-m + h_1) - \gamma(-m) - \gamma(-m + h_1 - h_2))^2 - \\
	- \frac{1}{n - h_2} \sum_{m = 1}^{n - h_2 - 1} m (\gamma(m - h_1) + \gamma(m + h_2) - \gamma(m - h_1 + h_2) - \gamma(m))^2)
\end{multline*}

По определению семивариограммы, $ \gamma(-h) = \gamma(h) $, тогда
\begin{multline*}
	cov\{ 2 \tilde{\gamma}(h_1), 2 \tilde{\gamma}(h_2) \} = \frac{2}{n - h_1} (\sum_{m = 1 - n + h_2}^{n - h_1 - 1} (\gamma(m - h_2) + \gamma(m + h_1) - \gamma(m) - \gamma(m + h_1 - h_2))^2 - \\
	- \frac{2}{n - h_2} \sum_{m = 1}^{n - h_2 - 1} m (\gamma(m + h_2) + \gamma(m - h_1) - \gamma(m) - \gamma(m - h_1 + h_2))^2)
\end{multline*}

Найдём предел для случая $ h_1 > h_2 $:
\begin{multline*}
  % \begin{split}
		\lim_{n \to \infty} (n - h_2) \frac{2}{n - h_2}(\sum_{m = 1 - n + h_2}^{n - h_1 - 1} (\gamma(m - h_2) + \gamma(m + h_1) - \gamma(m) - \gamma(m + h_1 - h_2))^2 - \\
		- \frac{2}{n - h_1} \sum_{m = 1}^{n - h_1 - 1} m (\gamma(m - h_2) + \gamma(m + h_1) - \gamma(m) - \gamma(m + h_1 - h_2))^2) =  \\
		= \lim_{n \to \infty} 2(\sum_{m = 1 - n + h_2}^{n - h_1 - 1} (\gamma(m - h_2) + \gamma(m + h_1) - \gamma(m) - \gamma(m + h_1 - h_2))^2 - \\
		- \frac{2}{n - h_1} \sum_{m = 1}^{n - h_1 - 1} m (\gamma(m - h_2) + \gamma(m + h_1) - \gamma(m) - \gamma(m + h_1 - h_2))^2) = \\
		= 2 \sum_{m = -\infty}^{+\infty} (\gamma(m - h_2) + \gamma(m + h_1) - \gamma(m) - \gamma(m + h_1 - h_2))^2 - \\
		- 4 \lim_{n \to \infty} \frac{1}{n - h_1} \sum_{m = 1}^{n - h_1 - 1} m (\gamma(m - h_2) + \gamma(m + h_1) - \gamma(m) - \gamma(m + h_1 - h_2))^2
	% \end{split}
\end{multline*}

Из предположения теоремы \eqref{eq:var_abs} ряд $ \sum\limits_{m = -\infty}^{+\infty} \gamma(h) $, сходится абсолютно $ \gamma(h) $ следовательно стремится к $ 0 $, при $ n \to \infty $, быстрее, чем $ \frac{1}{n} $. Тогда
$$ \gamma(m - h_2) + \gamma(m + h_1) - \gamma(m) - \gamma(m + h_1 - h_2) $$
стремится к $ 0 $, при $ m \to \infty $, быстрее, чем $\frac{1}{m}$, значит
\begin{align*}
	\gamma(m - h_2) + \gamma(m + h_1) - \gamma(m) - \gamma(m + h_1 - h_2) \sim o(\frac{1}{m}) \\
	(\gamma(m - h_2) + \gamma(m + h_1) - \gamma(m) - \gamma(m + h_1 - h_2))^2 \sim o(\frac{1}{m^2})
\end{align*}

Тогда ряд $ \sum\limits_{m = 1}^{n - h_1 - 1} m (\gamma(m - h_2) + \gamma(m + h_1) - \gamma(m) - \gamma(m + h_1 - h_2))^2 $ сходится при $ n \to \infty $, а значит
\begin{equation*}
	\lim_{n \to \infty} \sum_{m = 1}^{n - h_1 - 1} m (\gamma(m - h_2) + \gamma(m + h_1) - \gamma(m) - \gamma(m + h_1 - h_2))^2 = 0 \quad \text{и}
\end{equation*}
\begin{align}
\nonumber
	\lim_{n \to \infty} (n - h_2) \frac{2}{n - h_2}(\sum_{m = 1 - n + h_2}^{n - h_1 - 1} (\gamma(m - h_2) + \gamma(m + h_1) - \gamma(m) - \gamma(m + h_1 - h_2))^2 - \\ \nonumber
	- \frac{2}{n - h_1} \sum_{m = 1}^{n - h_1 - 1} m (\gamma(m - h_2) + \gamma(m + h_1) - \gamma(m) - \gamma(m + h_1 - h_2))^2) = \\
	= 2 \sum_{m = -\infty}^{+\infty} \gamma(m - h_2) + \gamma(m + h_1) - \gamma(m) - \gamma(m + h_1 - h_2))^2
\label{eq:asyfirst}
\end{align}

Рассуждая аналогично, получаем предел при $ h_1 < h_2 $:
\begin{align}
\nonumber
	\lim_{n \to \infty} (n - h_1) \frac{2}{n - h_1} (\sum_{m = 1 - n + h_2}^{n - h_1 - 1} (\gamma(m - h_2) + \gamma(m + h_1) - \gamma(m) - \gamma(m + h_1 - h_2))^2 - \\ \nonumber
	- \frac{2}{n - h_2} \sum_{m = 1}^{n - h_2 - 1} m (\gamma(m + h_2) + \gamma(m - h_1) - \gamma(m) - \gamma(m - h_1 + h_2))^2) = \\
	= 2 \sum_{m = -\infty}^{+\infty} \gamma(m - h_2) + \gamma(m + h_1) - \gamma(m) - \gamma(m + h_1 - h_2))^2
\label{eq:asysecond}
\end{align}

Тогда, объединяя вместе \eqref{eq:asyfirst} и \eqref{eq:asysecond}, получаем:
\begin{equation*}
	\lim_{n \to \infty} (n - \min\{ h_1, h_2 \}) cov\{ 2 \tilde{\gamma}(h_1), 2 \tilde{\gamma}(h_2) \} = 2 \sum_{m = -\infty}^{+\infty} \gamma(m - h_2) + \gamma(m + h_1) - \gamma(m) - \gamma(m + h_1 - h_2))^2.
\end{equation*}

Нетрудно видеть, что если положить $ h_1 = h_2 = h $, то
\begin{equation*}
	\lim_{n \to \infty} (n - h) V\{ 2 \tilde{\gamma}(h) \} = 2 \sum_{m = -\infty}^{+\infty} \gamma(m - h) + \gamma(m + h) - 2 \gamma(m))^2.
\end{equation*}

\end{proof}
% section new_section (end)