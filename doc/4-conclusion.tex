%!TEX root = thesis.tex

\chapter*{Заключение}
\addcontentsline{toc}{chapter}{Заключение}

В дипломной работе исследована температура воды озера Баторино (Беларусь). Исследование проводилось на основе данных, полученных от учебно-научного центра <<Нарочанская биологическая станция им. Г.Г.Винберга>>, состоящих из наблюдений, в период с 1975 по 2012 год в июле месяце.

Были вычислены и проанализированы описательные статистики, проведена проверка на нормальность. Обнаружено, что распределение температуры воды в озере Баторино близко к нормальному закону распределения с параметрами $ \mathcal{N}(20.08, 5.24) $. Отклонение от нормальности характеризуется полученными коэффициентами асимметрии и эксцесса, которые свидетельствуют о небольшой скошенности вправо и более пологой колоколообразной форме относительно нормального закона распределения. В результате проведённого корреляционного анализа вычислен коэффициент корреляции и доказана его значимость. Выявлена умеренная прямая зависимость между температурой воды и временем.

Проведён регрессионный анализ, в процессе которого была построена аддитивная модель временного ряда и найдён линейный тренд. Вычислен коэффициент детерминации, проверена значимость регрессионных коэффициентов критерием Стьюдента и адекватность модели критерием Фишера. Построенная детерминированными методами линейная регрессионная модель оказалась значимой и адекватной, но при этом описывает поведение временного ряда лишь частично. Что говорит о наличии некоторых неучтённых данной моделью факторов.

Как следствие удаления тренда из исходных данных получен ряд остатков. В результате его анализа была также показана близость распределения к нормальному. С помощью проверки тестов Дики-Фуллера и Льюнга-Бокса показана стационарность в широком смысле и отсутствие автокорреляций в ряде остатков.

Использованы также современные геостатистические методы анализа реального временного ряда. Исследованы свойства оценки вариограммы гауссовского случайного процесса: найдены первые два момента, изучено асимптотическое поведение моментов второго порядка. Доказана несмещённость и состоятельность в среднеквадратическом смысле оценки вариограммы.

В рамках вариограммного анализа были изучены оценки семивариограммы Матерона и Кресси-Хокинса, исследованы различные модели семивариограмм, рассмотрены два подхода по оценке качества модели --- перекрёстный и адаптивный, и два метода по подбору моделей и параметров --- визуальный и автоматический. Таким образом, наилучшими моделями для семивариограммы временного ряда, представляющего собой ряд остатков, оказались: линейная с порогом $ \widehat{\gamma}_3(h) $ вида \eqref{eq:linsill} и периодическая $\widehat{\gamma}_6(t) $ вида \eqref{eq:per}. Показаны преимущества по использованию перекрёстного метода подбора параметров для описания исследуемых данных и адаптивного метода для вычисления прогнозных значений. На основе построенных моделей семивариограмм с помощью интерполяционного метода кригинг вычислены прогнозные значения временного ряда. Изучена зависимость точности прогноза от оценки вариограммы и модели.

В процессе исследования показано, что в рассматриваемых условиях результаты, полученные автоматическим методом подбора моделей, оказались хуже, чем полученные визуальным подбором. Но при этом, автоматический подбор моделей можно использовать без глубоких знаний в вариограммном анализе, тогда как для визуального подбора требуется применение знаний по статистическому анализу временных рядов и знаний геостатистических методов. Сделаны выводы о возможности использования каждого из них: автоматический подбор следует использовать в случаях, когда нужно быстро получить результат и когда данные имеют плавный характер изменений. В остальных случаях, предпочтительнее использовать визуальный подбор параметров с учётом всех характерных особенностей исследуемых данных.

Результаты дипломной работы получены в среде программирования \textbf{R}. В частности, на её основе было реализованно программное обеспечение, позволяющее решать не только данную задачу, но и более широкий класс аналогичных задач в таких областях, как биология, гидрология, природопользование, экология и других. Разработанное приложение имеет простой, быстро расширяемый и гибкий интерфейс, поэтому может быть изменено и дополнено с учётом потребностей конкретного исследования. Для ознакомления, приложение доступно в сети интернет по адресу <<\textit{apaulau.shinyapps.io/batorino}>>.