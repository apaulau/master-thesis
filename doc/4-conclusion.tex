%!TEX root = thesis.tex

\chapter*{Заключение}
\addcontentsline{toc}{chapter}{Заключение}

В представленной работе были рассмотрены возможности по анализу случайных процессов с помощью языка программирования \textbf{R}. В частности было подробно рассмотрено реализованное на его основе программное обеспечение, позволяющее решать класс задач, аналогичных исследуемой.

В рамках этой работы, с помощью реализованного приложения была исследована важнейшая характеристика любого водоёма --- температура воды. Исследование проводилось на основе данных, полученных из наблюдений за озером Баторино, в период с 1975 по 2012 год в июле месяце. Для этого были вычислены и проанализированы описательные статистики, проведена проверка на нормальность, проведён визуальный анализ. В результате указанной части работы было обнаружено, что распределение температуры воды в озере Баторино близко к нормальному закону распределения с параметрами $\mathcal{N}(20.08, 5.24)$. Отклонение от нормальности отмечается полученными коэффициентами асимметрии и эксцесса. Исследуемое распределение имеет небольшую скошенность вправо и более растянутую колоколообразную форму относительно нормального закона распределения. В результате проведённого корреляционного анализа была выявлена умеренная зависимость между температурой воды и временем: был обнаружен рост температуры с течением времени.

В работе был проведён регрессионный анализ, в процессе которого была построена аддитивная модель временного ряда, найдён тренд, и, как следствие удаления тренда из построенной модели, был получен ряд остатков. Построенная детерминированными методами линейная регрессионная модель оказалась значимой и адекватной, но при этом описывает поведение временного ряда лишь частично. В результате анализа ряда остатков было выявлено отклонение распределения от нормальности. Что говорит о наличии некоторых неучтённых данной моделью факторов, затрудняющих дальнейшее исследование классическими методами. Следует также отметить стационарность и отсутствие автокорреляций в ряде остатков. Эти результаты говорят о постоянстве вероятностных свойств с течением времени, а также об отсутствии зависимостей между наблюдениями. Данные выводы позволили использовать современные геостатистические методы анализа.

В процессе вариограммного анализа были рассмотрены различные модели вариограмм, оценены свойства и поведение при изменении одного из параметров. Используя методы кригинга, на основе найденных моделей вычислены прогнозные значения и дана их оценка. С помощью применения кросс-валидации найдены наилучшие модели исходных данных.