%!TEX root = thesis.tex

\chapter*{Заключение}
\addcontentsline{toc}{chapter}{Заключение}

В представленной работе были рассмотрены возможности по анализу случайных процессов с помощью языка программирования \textbf{R}. В частности, на его основе было реализованно программное обеспечение, позволяющее решать класс задач, аналогичных исследуемой.

В рамках этой работы, с помощью реализованного приложения была исследована температура воды озера Баторино (Беларусь). Исследование проводилось на основе данных, полученных из наблюдений, в период с 1975 по 2012 год в июле месяце. Были вычислены и проанализированы описательные статистики, проведена проверка на нормальность. В результате указанной части работы было обнаружено, что распределение температуры воды в озере Баторино близко к нормальному закону распределения с параметрами $\mathcal{N}(20.08, 5.24)$. Отклонение от нормальности характеризуется полученными коэффициентами асимметрии и эксцесса. Исследуемое распределение имеет небольшую скошенность вправо и более растянутую колоколообразную форму относительно нормального закона распределения. В результате проведённого корреляционного анализа вычислен коэффициент корреляции ($ r_{xt} = \characteristic{original}{correlation} $) и доказана его значимость. Таким образом выявлена умеренная прямая зависимость между температурой воды и временем.

В работе проведён регрессионный анализ, в процессе которого была построена аддитивная модель временного ряда и найдён линейный тренд. Как следствие удаления тренда из построенной модели, был получен ряд остатков. В результате его анализа было также показана близость распределения к нормальному с чуть большими отклонениями, чем исходное. Что говорит о наличии некоторых неучтённых данной моделью факторов. Построенная детерминированными методами линейная регрессионная модель оказалась значимой и адекватной, но при этом описывает поведение временного ряда лишь частично. С помощью проведения ряда тестов показана стационарность и отсутствие автокорреляций в ряде остатков. Данные выводы позволили использовать современные геостатистические методы анализа.

Исследованы свойства оценки вариограммы гауссовского случайного процесса: найдены первые два момента, показано асимптотическое поведение моментов второго порядка. Как следствие, доказана несмещённость и состоятельность в среднеквадратическом смысле оценки вариограммы.

В результате вариограммного анализа были изучены оценки Матерона и Кресси-Хокинса, исследованы различные модели семивариограмм, рассмотрены два подхода по оценке качества модели --- перекрёстный и адаптивный, и два метода по подбору моделей и параметров --- визуальный и автоматический. Также построены наилучшие модели семивариограмм: линейная с порогом $ \widehat{\gamma}_3(h) $ вида \eqref{eq:linsill} и периодическая $\widehat{\gamma}_6(t) $ вида \eqref{eq:per}. Показаны преимущества по использованию перекрёстного метода подбора параметров для вычисления прогнозных значений и адаптивного метода для описания исследуемых данных. На основе построенных моделей семивариограмм вычислены прогнозные значения исследуемого временного ряда с помощью интерполяционного метода кригинг. Исследована зависимость точности прогноза от оценки вариограммы и модели.

В процессе исследования показано, что автоматический метод подбора моделей в рассматриваемых условиях показал результаты хуже, чем визуальный подбор. Но при этом, автоматический подбор моделей можно использовать без глубоких знаний в вариограммном анализе, тогда как для визуального подбора требуется применение знаний по статистическому анализу временных рядов и знаний геостатистических методов. Таким образом, сделаны выводы по случаям использования каждого из них: автоматический подбор следует использовать в случаях, когда нужно быстро получить результат и когда данные имеют плавных характер изменений. В остальных случаях, предпочтительнее использовать визуальный подбор параметров с учётом всех характерных особенностей исследуемых данных.