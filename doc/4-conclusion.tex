%!TEX root = thesis.tex

\chapter*{Заключение}
\addcontentsline{toc}{chapter}{Заключение}

В представленной работе были рассмотрены возможности по анализу случайных процессов с помощью языка программирования \textbf{R}. В частности было подробно рассмотрено реализованное на его основе программное обеспечение, позволяющее решать класс задач, аналогичных исследуемой.

В рамках этой работы, с помощью реализованного приложения была исследована важнейшая характеристика --- температура воды. Исследование проводилось на основе данных, полученных из наблюдений за озером Баторино, в период с 1975 по 2012 год в июле месяце. Для этого были вычислены и проанализированы описательные статистики, проведена проверка на нормальность, проведён визуальный анализ. В результате указанной части работы было обнаружено, что распределение температуры воды в озере Баторино близко к нормальному закону распределения с параметрами $\mathcal{N}(20.08, 5.24)$. Отклонение от нормальности характеризуется полученными коэффициентами асимметрии и эксцесса. Исследуемое распределение имеет небольшую скошенность вправо и более растянутую колоколообразную форму относительно нормального закона распределения. В результате проведённого корреляционного анализа вычислен коэффициент корреляции ($ r_{xt} = \characteristic{original}{correlation} $) и доказана его значимость. Таким образом выявлена умеренная зависимость между температурой воды и временем.

В работе проведён регрессионный анализ, в процессе которого была построена аддитивная модель временного ряда, найдён тренд, и, как следствие удаления тренда из построенной модели, был получен ряд остатков. В его результате анализа было также показана близость распределения к нормальному с чуть большими отклонениями, чем исходное. Что говорит о наличии некоторых неучтённых данной моделью факторов. Построенная детерминированными методами линейная регрессионная модель оказалась значимой и адекватной, но при этом описывает поведение временного ряда лишь частично. С помощью проведения ряда тестов показана стационарность и отсутствие автокорреляций в ряде остатков. Эти результаты говорят о постоянстве вероятностных свойств, а также об отсутствии зависимостей между наблюдениями. Данные выводы позволили использовать современные геостатистические методы анализа.

В результате вариограммного анализа были исследованы различные модели семивариограмм, проанализированы оценки Матерона и Кресси-Хокинса, исследованы два подхода по оценке качества модели и два метода по подбору моделей и параметров, а также построены модели, позволившие вычислить прогнозные значения. Даны рекомендации по использованию перекрёстного метод подбора параметров для вычисления прогнозных значений и адаптивного метода для описания исследуемых данных. В процессе анализа показано, что автоматический метод подбора моделей в рассматриваемых условиях показал результаты хуже, чем визуальный подбор. Но при этом, автоматический подбор моделей можно использовать без глубоких знаний о вариограммном анализе. Таким образом, сделаны выводы по случаям использования каждого из них: автоматический подбор следует использовать в случаях, когда нужно быстро получить результат и когда данные имеют плавных характер изменений. В остальных случаях, предпочтительнее использовать визуальный подбор параметров с учётом всех характерных особенностей исследуемых данных.