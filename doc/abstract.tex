%!TEX root = thesis.tex

\newpage

\chapter*{Реферат}
Дипломная работа 35 страниц, 3 главы, 11 рисунков, 7 таблиц, 29 источников, 4 приложения

% Ключевые слова
ВРЕМЕННЫЕ РЯДЫ, R, ОПИСАТЕЛЬНЫЕ СТАТИСТИКИ, КОРРЕЛЯЦИОННЫЙ АНАЛИЗ, РЕГРЕССИОННЫЙ АНАЛИЗ, АНАЛИЗ ОСТАТКОВ, ВАРИОГРАММА, КРИГИНГ.

\textit{Объектом исследования} являются наблюдения температуры воды в озере Баторино в период с 1975 по 2012 гг.

\textit{Цель работы} --- анализ, обработка и прогнозирование в современном пакете прикладных программ для статистического анализа R.

В процессе работы проведён сравнительный анализ современных пакетов статистического анализа. При помощи пакета R вычислены и проанализированы описательные статистики, произведена подборка закона распределения, проведёны корреляционный и регрессионный анализы, проанализирован ряд остатков, построены модели вариограмм и на их основе вычислены прогнозные значения.

Полученные результаты могут быть использованы для дальнейшего исследований в различных прикладных областях науки: биологии, химии, гидрологии, --- а также, для анализа экологической ситуации в Нарочанском парке и других регионах.

Данная работа может быть продолжена для получения модели, более точно описывающей поведение исходного временного ряда. Полученные в процесса работы алгоритмы исследования могут быть использованы для анализа других аналогичных данных.

\newpage

\chapter*{Abstract}
Diploma thesis, 35 pages, 3 chapters, 11 figures, 7 tables, 29 sources, 4 appendixes.

TIME SERIES, R, DISCRIPTIONAL STATISTICS, CORRELATIONAL ANALYSIS, REGRESSION ANALYSIS, RESIDUAL ANALYSIS, VARIOGRAMM, KRIGING.

\textit{Object of research} is water temperature observations of Batorino lake in period from 1975 till 2012.

\textit{Research purpose} --- analysis, processing and forecasting in modern software package for statistical analysis --- R.

During the research was performed comparative analysis of modern packages for statistical research. With help of R package were computed and analysed descriptional statistics, was performed destribution analysis and fitting, were conducted correlational and regression analysis, was performed analysis of residual time series, variogram models and based on them prediction values were computed.

Results of this research could be used for further researches in various applied areas of science: biology, chemistry, hydrology, --- and also for analysis of ecology situation at the Narochansky park and other regions.

This research could be continued in case of getting model that will be more accurate in describing source time series. Algorythms that were obtained during the research could be used for analysis other similar data.