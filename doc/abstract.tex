\documentclass[a4paper,14pt]{extreport}
% Шрифты для отображения кириллических символов
\usepackage[T1,T2A]{fontenc}
% Входная кодировка utf-8
\usepackage[utf8]{inputenc}
\usepackage{bsu/style/bsumain}
\usepackage{bsu/style/bsuabstracttitle}

\subfaculty{Кафедра теории вероятностей и математической статистики}
\title{Анализ и прогнозирование гидрологических данных}
\author{Павлов Александр Сергеевич}
\mentor{доцент кафедры ТВиМС, канд. физ.-мат. наук  \\
	 	Цеховая Татьяна Вячеславовна
}

\begin{document}

% \maketitle

\newpage

\chapter*{Реферат}
Дипломная работа, 57 страниц, 20 рисунков, 8 таблиц, 35 источников, 4 приложения

% Ключевые слова
ВРЕМЕННОЙ РЯД, ПРОГНОЗИРОВАНИЕ, R, ОПИСАТЕЛЬНЫЕ СТАТИСТИКИ, КОРРЕЛЯЦИОННЫЙ АНАЛИЗ, РЕГРЕССИОННЫЙ АНАЛИЗ, ВАРИОГРАММА, КРИГИНГ, КРОСС-ВАЛИДАЦИЯ.

\textit{Объектом исследования} являются наблюдения за температурой воды в озере Баторино в период с 1975 по 2012 гг.

\textit{Цель работы} --- с помошью современного языка программирования R осуществить анализ, обработку и прогнозирование реального временного ряда.

В процессе работы реализовано веб-приложение, позволяющее решать класс аналогичных поставленной задач. В данном приложении вычислены и проанализированы описательные статистики, подобран закона распределения, проведены корреляционный и регрессионный анализы, исследован ряд остатков, построены и проанализированы различные модели вариограмм и на их основе вычислены прогнозные значения временного ряда.

Полученные результаты могут быть использованы для дальнейших исследований в различных прикладных областях науки: биологии, химии, гидрологии, --- а также, для анализа экологической ситуации в Нарочанском парке и других регионах.

Реализованное программное обеспечение и предложенные алгоритмы могут использоваться для решения задач, аналогичных рассматриваемой в работе.

Данная работа может быть продолжена для получения модели, более точно описывающей поведение исходного временного ряда. Программное обеспечение и алгоритмы могут быть усовершенствованы в процессе дальнейших исследований, для решения других задач.

\newpage

\chapter*{Рэферат}
Дыпломная работа, 57 старонак, 20 малюнкаў, 8 таблiц, 35 крынiц, 4 прыкладання

ЧАСОВЫ ШЭРАГ, ПРАГНАЗАВАННЕ, R, АПІСАЛЬНЫЯ СТАТЫСТЫКI, КАРЭЛЯЦЫЙНЫ АНАЛIЗ, РЭГРЭСIЙЫ АНАЛIЗ, ВАРЫЯГРАММА, КРЫГIНГ, КРОСС-ВАЛIДАЦЫЯ

\textit{Аб'ектам даследавання} з'яўляюцца назіранні за тэмпературай вады ў возеры Баторына ў перыяд з 1975 па 2012 гг.

\textit{Мэта працы} --- з дапамогай сучаснай мовы праграмавання R ажыццявіць аналіз і прагназаванне рэальнага часовага шэрага.

У працэсе працы рэалізаваны вэб-дадатак, якое дазваляе вырашаць клас аналагічных пастаўленай задач. У дадзеным дадатку вылічаны і прааналізаваны апісальныя статыстыкі, падабраны закон размеркавання, праведзен карэляцыйны і рэгрэсійны аналізы, даследаван шэраг рэшткаў, пабудаваны і прааналізаваны розныя мадэлі варыяграмм і на іх аснове вылічаны прагнозныя значэння часовага шэрага.

Атрыманыя вынікі могуць быць выкарыстаны для далейшых даследаванняў у розных прыкладных галінах навукі: біялогіі, хіміі, гідралогіі, --- а таксама, для аналізу экалагічнай сітуацыі ў Нарачанскім парку і іншых рэгіёнах.

Рэалізаванае праграмнае забеспячэнне і прапанаваныя алгарытмы могуць выкарыстоўвацца для вырашэння задач, аналагічных разгледзенай у працы.

Дадзеная праца можа быць працягнутая для атрымання мадэлі, больш дакладна апісвае паводзіны зыходнага часовага шэрагу. Праграмнае забеспячэнне і алгарытмы могуць быць удасканалены ў працэсе далейшых даследаванняў, для вырашэння іншых задач.

\newpage

\chapter*{Abstract}
Bachelor's thesis, 57 pages, 20 figures, 8 tables, 35 sources, 4 appendices.

TIME SERIES, PREDICTION, R, DESCRIPTIONAL STATISTICS, CORRELATIONAL ANALYSIS, REGRESSION ANALYSIS, VARIOGRAMM, KRIGING, CROSS-VALIDATION.

\textit{Object of research} is water temperature observations of Batorino lake in period from 1975 till 2012.

\textit{Research purpose} --- with help of modern programming language R perform analysis and forecasting real time series.

During the research was implemented rich web-application that allows to solve problems similar to researched within current thesis. With help of implemented application and R programming language were computed and analysed descriptional statistics, was performed destribution analysis and fitting, were conducted correlation and regression analysis, was performed research of residual time series, variogram models and based on them time series prediction values were computed.

Results of this research could be used for further researches in various applied areas of science: biology, chemistry, hydrology, --- and also for analysis of ecology situation at the Narochansky park and other regions.

Implemented web-application and suggested algorythms could be used in case of solving problems with similar with this research problem.

This research could be continued in case of getting model that will be more accurate in describing source time series. Software and algorythms that were obtained during the research could be enhanced during further research for solving different problems.

\end{document}