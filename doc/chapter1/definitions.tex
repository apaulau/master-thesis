% (PARTIAL)
\newpage
\chapter{Случайный процесс и его характеристики. Стационарность случайных процессов. Вариограмма}
\label{c:definitions}

\section{Случайный процесс. Стационарность}

Для введения следующих понятий воспользуемся \cite{brillinjer-ts, trush-ts}.

Пусть $ (\Omega, \mathcal{F}, P) $ --- вероятностное пространство, где $\Omega$ является произвольным множеством, $\mathcal{F}$ --- сигма-алгеброй подмножеств $\Omega$, и $P$ --- вероятностной мерой.

%TODO
\begin{Definition}\label{def:stochasticProcess}
Попробуй сделать через дефинишены, должно получаться красивее и можно будет на них ссылаться
\end{Definition}

\textit{Действительным случайным процессом} $ X(t) = X(\omega, t) $ называется семейство случайных величин, заданных на вероятностном пространстве $ (\Omega, \mathcal{F}, P) $, где $ \omega \in \Omega, t \in \mathbb{T}$, где $ \mathbb{T} $ --- некоторое параметрическое множество.

Если $ \mathbb{T} = \mathbb{Z} = {0, \pm 1, \pm 2, \dots} $, или $ \mathbb{T} \subset \mathbb{Z} $, то говорят, что $ X(t), t \in \mathbb{T} $, --- \textit{случайный процесс с дискретным временем}.

Если $ \mathbb{T} = \mathbb{R} $, то $ X(t), t \in \mathbb{T} $ называют \textit{случайным процессом с непрерывным временем}.

\textit{n-мерной функцией распределения} случайного процесса $ X(t), t \in \mathbb{T} $, называется функция вида
\begin{equation*}
	F_n(x_1, \dots, x_n; t_1, \dots, t_n) = P \{ X(t_1) < x_1, \dots, X(t_n) < x_n \},
\end{equation*}
где $ x_j \in \mathbb{T}, t_j \in \mathbb{T}, j = \overline{1,n} $.

\textit{Математическим ожиданием} случайного процесса $ X(t), t \in \mathbb{T}, $ называется функция вида
\begin{equation*}
	m(t) = E \{ X(t) \} = \int \limits_{\mathbb{T}} x \, dF_1(x;t), t \in \mathbb{T}.
\end{equation*}

\textit{Дисперсией} случайного процесса $ X(t), t \in \mathbb{T} $ называется функция вида:
\begin{equation*}
	V(t) = V \{ X(t) \} = E \{ X(t) - m(t) \}^2 = \int \limits_{\mathbb{T}} (x - m(t))^2 \, dF_1(x; t).
\end{equation*}

%TODO поменяй местами определение ковариационной и корреляционной функции. и сразу же напиши про связь корр=ков/корень из произведения дисперсий

\textit{Корреляционной функцией} случайного процесса $ X(t), t \in \mathbb{T} $ называется функция вида:
\begin{equation*}
	corr(X(t_1), X(t_2)) = E \{ X(t_1)X(t_2) \} = \iint \limits_{\mathbb{T}^2} x_1 x_2 \, dF_2(x_1, x_2; t_1, t_2)
\end{equation*}

\textit{Ковариационной функцией} случайного процесса $ X(t), t \in \mathbb{T} $ называется функция вида:
\begin{eqnarray*}
	& cov(X(t_1), X(t_2)) = E \{ (X(t_1) - m(t_1)) (X(t_2) - m(t_2)) \} = \\
	& = \iint \limits_{\mathbb{T}^2} (x_1 - m(t_1)) (x_2 - m(t_2)) \, dF_2(x_1, x_2; t_1, t_2)
\end{eqnarray*}

%TODO Сначала напиши про стационарность в гироком смысле, потом в узком, и замечание про связь этих определений (дополнительное условие на ковариационную функцию у стационарного в кзком смысле СП

Случайный процесс $ X(t), t \in \mathbb{T} $, называется стационарным в узком смысле, если $ \forall n \in \mathbb{N} $, $ \forall t_1, \dots, t_n \in \mathbb{T} $, $ \forall \tau, t_1 + \tau, \dots, t_n + \tau \in \mathbb{T} $ выполняется соотношение:
\begin{equation*}
	F_n(x_1, \dots, x_n; t_1, \dots, t_n) = F_n(x_1, \dots, x_n; t_1 + \tau , \dots, t_n + \tau).
\end{equation*}

Случайный процесс $X (t), t \in \mathbb{T} $, называется стационарным в широком смысле, если $ \exists E \{ x^2(t) < \infty \}, t \in \mathbb{T} $, и
\begin{enumerate}
	\item $ m(t) = E \{ x(t) \} = m = const, t \in \mathbb{T} $;
	\item $ cov(t_1, t_2) = cov(t_1 - t_2), t_1,t_2 \in \mathbb{T} $.
\end{enumerate}

\begin{Remark}
	Если случайный процесс $ X(t), t \in \mathbb{T} $, является стационарным в узком смысле и $ \exists E \{ x^2(t) \} < \infty, t \in \mathbb{T} $, то он будет стационарным и в широком смысле, но не наоборот.
\end{Remark}

\section{Вариограмма и внутренне стационарный случайный процесс}
\label{sec:variogramAndInnerStationarity}

%TODO Дай определение вариограммы и семивариограммы отдельно

%TODO определение внутренне стационарного соучайного процесса дай для процесса с непрерывным временем, можно взять из статьи брест 2005

Случайный процесс $ X(t), t \in \mathbb{Z} = \{0, \pm 1, \pm 2, \dots \} $, называется внутренне стационарным, если справедливы следующие равенства:
\begin{equation*}
	E \{ X(t_1) - X(t_2) \} = 0, \quad V \{ X(t_1) - X(t_2) \} = 2 \gamma(t_1 - t_2),
\end{equation*}
где $ 2 \gamma(t_1 - t_2) $ --- вариограмма рассматриваемого процесса, $ t_1, t_2 \in \mathbb{Z} $.

%TODO дай определение гацссовского случ процесса (у которого любая совсестная фр отсчетов является нормальной фр

в дальнейшем рассматриваем случайные процессы с дискретным временем.

Пусть $ X(t), t \in \mathbb{Z} $ --- внутренне стационарный гауссовский случайный процесс с нулевым математическим ожиданием, дисперсией $ \sigma^2 $ и неизвестной вариограммой.
\begin{equation*}
	2 \gamma(h) = V \{ X(t + h) - X(t) \}, ~ t,h \in \mathbb{Z}.
\end{equation*}