% (PARTIAL)
\newpage
\chapter{Определения и вспомогательные результаты}
\label{c:theory}

Пусть $ (\Omega, \mathcal{F}, P) $ --- вероятностное пространство, где $\Omega$ является произвольным множеством, $\mathcal{F}$ --- сигма-алгеброй подмножеств $\Omega$, и $P$ --- вероятностной мерой.

\textit{Действительным случайным процессом} $ X(t) = X(\omega, t) $ называется семейство случайных величин, заданных на вероятностном пространстве $ (\Omega, \mathcal{F}, P) $, где $ \omega \in \Omega, t \in \mathbb{R}$.

\textit{n-мерной функцией распределения} случайного процесса $ X(t), t \in \mathbb{R} $, называется функция вида
\begin{equation*}
	F_n(x_1, \dots, x_n; t_1, \dots, t_n) = P \{ X(t_1) < x_1, \dots, X(t_n) < x_n \},
\end{equation*}
где $ x_j \in \mathbb{R}, t_j \in \mathbb{R}, j = \overline{1,n} $.

\textit{Математическим ожиданием} случайного процесса $ X(t), t \in \mathbb{R}, $ называется функция вида
\begin{equation*}
	m(t) = E \{ X(t) \} = \int \limits_{\mathbb{R}} x \, dF_1(x;t), t \in \mathbb{R}.
\end{equation*}

\textit{Дисперсией} случайного процесса $ X(t), t \in \mathbb{R} $ называется функция вида:
\begin{equation*}
	V(t) = V \{ X(t) \} = E \{ X(t) - m(t) \}^2 = \int \limits_{\mathbb{R}} (x - m(t))^2 \, dF_1(x; t).
\end{equation*}

\textit{Корреляционной функцией} случайного процесса $ X(t), t \in \mathbb{R} $ называется функция вида:
\begin{equation*}
	corr(X(t_1), X(t_2)) = E \{ X(t_1)X(t_2) \} = \iint \limits_{\mathbb{R}^2} x_1 x_2 \, dF_2(x_1, x_2; t_1, t_2)
\end{equation*}

\textit{Ковариационной функцией} случайного процесса $ X(t), t \in \mathbb{R} $ называется функция вида:
\begin{equation*}
	cov(X(t_1), X(t_2)) = E \{ (X(t_1) - m(t_1)) (X(t_2) - m(t_2)) \} = \iint \limits_{\mathbb{R}^2} (x_1 - m(t_1)) (x_2 - m(t_2)) \, dF_2(x_1, x_2; t_1, t_2)
\end{equation*}

Случайный процесс $ X(t), t \in \mathbb{R} $, называется стационарным в узком смысле, если $ \forall n in \mathbb{N} $, $ \forall t_1, \dots, t_n \in \mathbb{R} $, $ \forall \tau, t_1 + \tau, \dots, t_n + \tau \in \mathbb{R} $

Случайный процесс $ X(s), s \in \mathbb{Z} = \{0, \pm 1, \pm 2, \dots \} $, называется внутренне стационарным, если справедливы следующие равенства:
\begin{equation*}
	E \{ X(s_1) - X(s_2) \} = 0, \quad V \{ X(s_1) - X(s_2) \} = 2 \gamma(s_1 - s_2),
\end{equation*}
где $2 \gamma(s_1 - s_2)$ --- вариограмма рассматриваемого процесса, $s_1,s_2 \in \mathbb{Z}$.

Пусть $ X(s), s \in \mathbb{Z} $ --- внутренне стационарный гауссовский случайный процесс с нулевым математическим ожиданием, дисперсией $ \sigma^2 $ и неизвестной вариограммой.
\begin{equation*}
	2 \gamma(h) = V \{ X(s+h) - X(s) \}, ~ s,h \in \mathbb{Z}.
\end{equation*}

\pgfplotsset{domain=-1:1}
\begin{tikzpicture}[baseline]
\begin{axis}[
    title = $h_1 > h_2$,
    xlabel = {$s$},
    ylabel = {$m$},
    xmin=-1.5, xmax=9.5,
    ymin=-5, ymax=3,
	axis x line=middle,
	axis y line=middle,
    xticklabels = {0, , , , , $n - h_1$},
    yticklabels = { ,$1 - n + h_2$, $h_2 - h_1$, 0, $n - h_1 -1$}
]
	\addplot coordinates {
		(1, 0)
		(8, 2)
		(8, 0)
		(8, -2)
		(1, -4)
		(1, 0)
	};
	\addplot[dashed, draw=gray, domain=0:8]{-2};
	\addplot[dashed, draw=gray, domain=0:8]{2};
\end{axis}
\end{tikzpicture}
\hspace{0.15cm}
\begin{tikzpicture}[baseline]
\begin{axis}[
    title = $h_1 > h_2$,
    xlabel = {$s$},
    ylabel = {$m$},
    xmin=-1.5, xmax=9.5,
    ymin=-5, ymax=3,
	axis x line=middle,
	axis y line=middle,
    xticklabels = {0, , , , , $n - h_1$},
    yticklabels = { ,$1 - n + h_2$, $h_2 - h_1$, 0, $n - h_1 -1$}
]
	\addplot coordinates {
		(1, 0)
		(8, 2)
		(8, 0)
		(8, -2)
		(1, -4)
		(1, 0)
	};
	\addplot[dashed, draw=gray, domain=0:8]{-2};
	\addplot[dashed, draw=gray, domain=0:8]{2};
\end{axis}
\end{tikzpicture}