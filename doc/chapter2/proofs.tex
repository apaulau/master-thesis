\newpage

\chapter{?? Теория ??}
\label{c:theory}

\section{Оценка вариограммы гауссовского случайного процесса} % (fold)

В качестве оценки вариограммы рассмотрим статистику вида:
\begin{equation}
	\label{eq:var_est}
	2 \tilde{\gamma}(h) = \frac{1}{n-h} \sum_{s=1}^{n-h}(X(s+h) - X(s))^2, \quad h = \overline{0, n-1}.
\end{equation}

Вычислим математическое ожидание введённой оценки
\begin{eqnarray*}
	& E \{ 2 \tilde{\gamma}(h) \} = \frac{1}{n-h} \sum_{s=1}^{n-h} E(X(s+h) - X(s))^2 = \\
	& = [\text{так как процесс является внутренне стационарным}] = \\
	& = \frac{1}{n-h} \sum_{s=1}^{n-h} 2 \gamma(h) = 2 \gamma(h).
\end{eqnarray*}

Таким образом оценка является несмещённой.

Далее, найдём ковариацию:
\begin{eqnarray*}
	& cov(2 \tilde{\gamma}(h_1), 2 \tilde{\gamma}(h_2)) = E((2 \tilde{\gamma}(h_1) - E(2 \tilde{\gamma}(h_1))) (2 \tilde{\gamma}(h_2) - E(2 \tilde{\gamma}(h_2)))) = \\
	& = E(\frac{1}{n-h_1} \sum_{s=1}^{n-h_1}((x(s+h_1) - x(s))^2 - E((x(s+h_1) - x(s))^2)) \times \\
	& \times \frac{1}{n-h_2} \sum_{t=1}^{n-h_2}((x(t+h_2) - x(t))^2 - E((x(t+h_2) - x(t))^2))) = \\
	& = \frac{1}{(n-h_1)(n-h_2)}\sum_{s=1}^{n-h_1}\sum_{t=1}^{n-h_2} cov((x(s+h_1) - x(s))^2, (x(t+h_2) - x(t))^2) 
\end{eqnarray*}

По определению, $ cov(a,b) = corr(a,b)\sqrt{V(a)V(b)} $, тогда
\begin{eqnarray*}
	& cov((x(s+h_1) - x(s))^2, (x(t+h_2) - x(t))^2) = corr((x(s+h_1) - x(s))^2, (x(t+h_2) - x(t))^2) \times \\
	& \times \sqrt{(V((x(s+h_1) - x(s))^2) V((x(t+h_2) - x(t))^2))}
\end{eqnarray*}

Из [ссылка на источник, либо упоминание раньше] $ V \{ X(s+h) - X(s) \}^2 = 2 \{ 2 \gamma(h) \}^2 $ и предыдущего соотношения:
\begin{equation*}
	cov(2 \tilde{\gamma}(h_1), 2 \tilde{\gamma}(h_2)) = \frac{2 (2\gamma(h_1))(2\gamma(h_2))}{(n-h_1)(n-h_2)}\sum_{s=1}^{n-h_1}\sum_{t=1}^{n-h_2} corr((x(s+h_1) - x(s))^2, (x(t+h_2) - x(t))^2)
\end{equation*}

Далее воспользуемся леммой 1 [\#Брест2005]:
\begin{eqnarray*}
	& \frac{2 (2\gamma(h_1))(2\gamma(h_2))}{(n-h_1)(n-h_2)}\sum_{s=1}^{n-h_1}\sum_{t=1}^{n-h_2} corr((x(s+h_1) - x(s))^2, (x(t+h_2) - x(t))^2) = \\
	& = \frac{2 (2\gamma(h_1))(2\gamma(h_2))}{(n-h_1)(n-h_2)}\sum_{s=1}^{n-h_1}\sum_{t=1}^{n-h_2} \{corr((x(s+h_1) - x(s)), (x(t+h_2) - x(t))) \}^2 = \\
	& = \frac{2 (2\gamma(h_1))(2\gamma(h_2))}{(n-h_1)(n-h_2)}\sum_{s=1}^{n-h_1}\sum_{t=1}^{n-h_2} \{\frac{cov((x(s+h_1) - x(s)), (x(t+h_2) - x(t)))}{\sqrt{(V((x(s+h_1) - x(s))) V((x(t+h_2) - x(t))))}} \}^2
\end{eqnarray*}

Принимая во внимание лемму 3 [\#Брест2005], получаем соотношение
\begin{eqnarray}
\nonumber
\label{eq:cov_base}
	& cov(2 \tilde{\gamma}(h_1), 2 \tilde{\gamma}(h_2)) = \\
	& = \frac{2 (2\gamma(h_1))(2\gamma(h_2))}{(n-h_1)(n-h_2)}\sum_{s=1}^{n-h_1}\sum_{t=1}^{n-h_2} \{\frac{\gamma(s+h_1-t) + \gamma(s-t-h_2) - \gamma(s+h_1-t-h_2) - \gamma(s-t)}{\sqrt{2 \gamma(h_1)} \sqrt{2 \gamma(h_2)}} \}^2
\end{eqnarray}

В \eqref{eq:cov_base} сделаем следующую замену переменных
\begin{equation*}
	s = s, \quad m = s - t.
\end{equation*}
Получим
\begin{eqnarray}
\nonumber
\label{eq:cov_split}
	& cov(2 \tilde{\gamma}(h_1), 2 \tilde{\gamma}(h_2)) = \frac{2(2 \gamma(h_1))(2 \gamma(h_2))}{(n - h_1) (n - h_2))} \sum_{s = 1}^{n - h_1}\sum_{s - m = 1}^{n - h_2} (\frac{\gamma(m + h_1) + \gamma(m - h_2) - \gamma(m + h_1 - h_2) - \gamma(m)}{\sqrt{2 \gamma(h_1)} \sqrt{2 \gamma(h_2)}})^2 = \\
	& = \frac{2}{(n - h_1) (n - h_2))} \sum_{s = 1}^{n - h_1}\sum_{s - m = 1}^{n - h_2} (\gamma(m + h_1) + \gamma(m - h_2) - \gamma(m + h_1 - h_2) - \gamma(m))^2
\end{eqnarray}

Таким образом, в зависимости от $h_1$ и $h_2$, возможны два случая: $h_1 > h_2$ и $h_2 > h_1$.

\begin{figure}[htbp]
	\centering
	\begin{tikzpicture}[baseline, font=\tiny]
		\begin{axis}[
		    title = $h_1 > h_2$,
		    xlabel = {$s$},
		    ylabel = {$m$},
		    xmin=-1.5, xmax=9.5,
		    ymin=-5, ymax=3,
			axis x line=middle,
			axis y line=middle,
		    xticklabels = {0, , , , , $n - h_1$},
		    yticklabels = { ,$1 - n + h_2$, $h_2 - h_1$, 0, $n - h_1 -1$},
		    x tick label style={anchor=north west},
		]
			\addplot coordinates {
				(1, 0)
				(8, 2)
				(8, 0)
				(8, -2)
				(1, -4)
				(1, 0)
			};
			\addplot[dashed, draw=gray, domain=0:8]{-2};
			\addplot[dashed, draw=gray, domain=0:8]{2};
			\node at (axis cs:3.5, 1.3) [rotate=17, anchor=north east] {$m = s - 1$};
			\node at (axis cs:4.5, -2.4) [rotate=19, anchor=north east] {$m = s - n + h_2$};
		\end{axis}
		\end{tikzpicture}
		\hspace{0.15cm}
		\begin{tikzpicture}[baseline, font=\tiny]
		\begin{axis}[
		    title = $h_1 < h_2$,
		    xlabel = {$s$},
		    ylabel = {$m$},
		    xmin=-1.5, xmax=9.5,
		    ymin=-3, ymax=5,
			axis x line=middle,
			axis y line=middle,
		    xticklabels = {0, , , , , $n - h_1$},
		    yticklabels = { ,$1 - n + h_2$, 0, $h_2 - h_1$, $n - h_1 - 1$}
		]
			\addplot coordinates {
				(1, 0)
				(8, 4)
				(8, 2)
				(1, -2)
				(1, 0)
			};
			\addplot[dashed, draw=gray, domain=0:8]{2};
			\addplot[dashed, draw=gray, domain=0:8]{4};
			\node at (axis cs:3.5, 2) [rotate=30, anchor=north east] {$m = s - 1$};
			\node at (axis cs:3.7, 0.2) [rotate=32, anchor=north east] {$m = s - n + h_2$};
		\end{axis}
	\end{tikzpicture}

	\caption{Замена переменных}
	\label{fig:label}
\end{figure}

Рассмотрим первый: $h_1 > h_2$. Поменяем порядок суммирования в \eqref{eq:cov_split}.
\begin{eqnarray*}
\nonumber
	& cov(2 \tilde{\gamma}(h_1), 2 \tilde{\gamma}(h_2)) = \\
	& = \frac{2}{(n - h_1) (n - h_2)} \sum_{s = 1}^{n - h_1}\sum_{s - m = 1}^{n - h_2} (\gamma(m + h_1) + \gamma(m - h_2) - \gamma(m + h_1 - h_2) - \gamma(m))^2 = \\
	& = \sum_{m = 1 - n + h_2}^{h_2 - h_1}\sum_{s = 1}^{m + n - h_2}(\gamma(m + h_1) + \gamma(m - h_2) - \gamma(m + h_1 - h_2) - \gamma(m))^2 + \\
	& + \sum_{m = h_2 - h_1 + 1}^{0}\sum_{s = 1}^{n - h_1}(\gamma(m + h_1) + \gamma(m - h_2) - \gamma(m + h_1 - h_2) - \gamma(m))^2 + \\
	& + \sum_{m = 1}^{n - h_1 - 1}\sum_{s = m + 1}^{n - h_1}(\gamma(m + h_1) + \gamma(m - h_2) - \gamma(m + h_1 - h_2) - \gamma(m))^2
\end{eqnarray*}

Заметим, что выражение под знаком суммы не зависит от $s$, получим:
\begin{eqnarray*}
\nonumber
	& cov(2 \tilde{\gamma}(h_1), 2 \tilde{\gamma}(h_2)) = \frac{2}{(n - h_1) (n - h_2)} \times \\
	& \times ((m + n - h_1)\sum_{m = 1 - n + h_2}^{h_2 - h_1}(\gamma(m + h_1) + \gamma(m - h_2) - \gamma(m + h_1 - h_2) - \gamma(m))^2 + \\
	& + (n - h_1)\sum_{m = h_2 - h_1 + 1}^{0}(\gamma(m + h_1) + \gamma(m - h_2) - \gamma(m + h_1 - h_2) - \gamma(m))^2 + \\
	& + (n - h_1 - m)\sum_{m = 1}^{n - h_1 - 1}(\gamma(m + h_1) + \gamma(m - h_2) - \gamma(m + h_1 - h_2) - \gamma(m))^2)
\end{eqnarray*}

Разделим каждое слагаемое на общий член $n - h_1$:
\begin{eqnarray*}
\nonumber
	& cov(2 \tilde{\gamma}(h_1), 2 \tilde{\gamma}(h_2)) = \\ 
	& = \frac{2}{(n - h_2)} ((1 + \frac{m}{n - h1})\sum_{m = 1 - n + h_2}^{h_2 - h_1}(\gamma(m + h_1) + \gamma(m - h_2) - \gamma(m + h_1 - h_2) - \gamma(m))^2 + \\
	& + \sum_{m = h_2 - h_1 + 1}^{0}(\gamma(m + h_1) + \gamma(m - h_2) - \gamma(m + h_1 - h_2) - \gamma(m))^2 + \\
	& + (1 - \frac{m}{n - h1})\sum_{m = 1}^{n - h_1 - 1}(\gamma(m + h_1) + \gamma(m - h_2) - \gamma(m + h_1 - h_2) - \gamma(m))^2)
\end{eqnarray*}

\label{sec:variogram_theory}

% section variogram_theory (end)