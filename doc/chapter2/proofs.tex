\newpage

\chapter{?? Теория ??}
\label{c:theory}

\section{Оценка вариограммы гауссовского случайного процесса} % (fold)
\label{sec:variogram_theory}

В качестве оценки вариограммы рассмотрим статистику вида:
\begin{equation}
	\label{eq:var_est}
	2 \tilde{\gamma}(h) = \frac{1}{n - h} \sum_{t = 1}^{n - h}(X(t + h) - X(t))^2, \quad h = \overline{0, n - 1},
\end{equation}
где $ \tilde{\gamma}(-h) = \tilde{\gamma}(h), h = \overline{0, n - 1}$; $ \tilde{\gamma}(h) = 0, |h| \ge n $.

Вычислим математическое ожидание введённой оценки
\begin{eqnarray*}
	& E \{ 2 \tilde{\gamma}(h) \} = \frac{1}{n - h} \sum_{t = 1}^{n - h} E \{ (X(t + h) - X(t))^2 \} = \\
	& = [\text{так как процесс является внутренне стационарным}] = \\ %TODO: change it
	& = \frac{1}{n - h} \sum_{t = 1}^{n - h} 2 \gamma(h) = 2 \gamma(h).
\end{eqnarray*}

Таким образом оценка является несмещённой.

Далее, найдём ковариацию:
\begin{eqnarray}
\label{eq:cov_support}
\nonumber
	& cov\{ 2 \tilde{\gamma}(h_1), 2 \tilde{\gamma}(h_2) \} = E\{ (2 \tilde{\gamma}(h_1) - E\{ 2 \tilde{\gamma}(h_1) \}) (2 \tilde{\gamma}(h_2) - E\{ 2 \tilde{\gamma}(h_2) \}) \} = \\
\nonumber
	& = E\{ \frac{1}{n - h_1} \sum_{t = 1}^{n - h_1}((X(t + h_1) - X(t))^2 - E\{ (X(t + h_1) - X(t))^2 \}) \times \\
\nonumber
	& \times \frac{1}{n - h_2} \sum_{s = 1}^{n - h_2}((X(s + h_2) - X(s))^2 - E\{ (X(s + h_2) - X(s))^2 \}) \} = \\
	& = \frac{1}{(n - h_1)(n - h_2)} \sum_{t = 1}^{n - h_1}\sum_{s = 1}^{n - h_2} cov\{ (X(t + h_1) - X(t))^2, (X(s + h_2) - X(s))^2 \}
\end{eqnarray}

По определению, $ cov\{ a, b \} = corr\{ a, b \} \sqrt{ V\{ a \} V\{ b \} } $, тогда
\begin{eqnarray*}
	& cov\{ (X(t + h_1) - X(t))^2, (X(s + h_2) - X(s))^2 \} = \\
	& = corr\{(X(t + h_1) - X(t))^2, (X(s + h_2) - X(s))^2 \} \times \\
	& \times \sqrt{V\{ (X( t + h_1) - X(t))^2 \} V\{ (X(s + h_2) - X(s))^2 \}}
\end{eqnarray*}

Принимая во внимание (\ref{eq:V_diff_inc}) и предыдущее соотношение, из (\ref{eq:cov_support}) получаем:
\begin{eqnarray*}
	& cov\{ (X(t + h_1) - X(t))^2, (X(s + h_2) - X(s))^2 \} = \\
	& = \frac{2 (2\gamma(h_1))(2\gamma(h_2))}{(n - h_1)(n - h_2)}\sum_{t = 1}^{n - h_1}\sum_{s = 1}^{n - h_2} corr\{(X(t + h_1) - X(t))^2, (X(s + h_2) - X(s))^2 \}
\end{eqnarray*}

Далее воспользуемся леммой 1 из \cite{tsekhavaya-brest}:
\begin{eqnarray*}
	& cov\{ 2 \tilde{\gamma}(h_1), 2 \tilde{\gamma}(h_2) \} = \\
	& = \frac{2 (2\gamma(h_1))(2\gamma(h_2))}{(n - h_1)(n - h_2)}\sum_{t = 1}^{n - h_1}\sum_{s = 1}^{n - h_2} (corr\{(X(t + h_1) - X(t))^2, (X(s + h_2) - X(s))^2 \})^2 = \\
	& = \frac{2 (2\gamma(h_1))(2\gamma(h_2))}{(n - h_1)(n - h_2)}\sum_{t = 1}^{n - h_1}\sum_{s = 1}^{n - h_2} ( \frac{cov\{ X(t + h_1) - X(t), X(s + h_2) - X(s) \}}{\sqrt{V\{ X( t + h_1) - X(t) \} V\{ X(s + h_2) - X(s) \}}} )^2
\end{eqnarray*}

Воспользовавшись леммой 3 из \cite{tsekhavaya-brest}, получаем соотношение
\begin{eqnarray}
\nonumber
\label{eq:cov_base}
	& cov\{ 2 \tilde{\gamma}(h_1), 2 \tilde{\gamma}(h_2) \} = \frac{2}{(n - h_1)(n - h_2)} \times \\
	& \times \sum_{t = 1}^{n - h_1}\sum_{s = 1}^{n - h_2} (\gamma(t - h_2 - s) + \gamma(t + h_1 - s) - \gamma(t - s) - \gamma(t + h_1 - s - h_2))^2
\end{eqnarray}

В \eqref{eq:cov_base} сделаем следующую замену переменных
\begin{equation*}
	t = t, \quad m = t - s.
\end{equation*}
Получим
\begin{eqnarray}
\nonumber
\label{eq:cov_split}
	& cov\{ 2 \tilde{\gamma}(h_1), 2 \tilde{\gamma}(h_2) \} = \\
	& = \frac{2}{(n - h_1) (n - h_2)} \sum_{t = 1}^{n - h_1}\sum_{t - m = 1}^{n - h_2} (\gamma(m - h_2) + \gamma(m + h_1) - \gamma(m) - \gamma(m + h_1 - h_2))^2
\end{eqnarray}

Таким образом, в зависимости от $h_1$ и $h_2$, возможны два случая: $h_1 > h_2$ и $h_1 < h_2$.

\begin{figure}[H]
	\centering
	\begin{tikzpicture}[baseline, font=\tiny]
		\begin{axis}[
		    title = $h_1 > h_2$,
		    xlabel = {$t$},
		    ylabel = {$m$},
		    xmin=-1.5, xmax=9.5,
		    ymin=-5, ymax=3,
			axis x line=middle,
			axis y line=middle,
		    xticklabels = {0, , , , , $n - h_1$},
		    yticklabels = { ,$1 - n + h_2$, $h_2 - h_1$, 0, $n - h_1 -1$},
		    x tick label style={anchor=north west},
		]
			\addplot coordinates {
				(1, 0)
				(8, 2)
				(8, 0)
				(8, -2)
				(1, -4)
				(1, 0)
			};
			\addplot[dashed, draw=gray, domain=0:8]{-2};
			\addplot[dashed, draw=gray, domain=0:8]{2};
			\node at (axis cs:3.5, 1.3) [rotate=17, anchor=north east] {$m = t - 1$};
			\node at (axis cs:4.5, -2.4) [rotate=19, anchor=north east] {$m = t - n + h_2$};
		\end{axis}
		\end{tikzpicture}
		\hspace{0.15cm}
		\begin{tikzpicture}[baseline, font=\tiny]
		\begin{axis}[
		    title = $h_1 < h_2$,
		    xlabel = {$t$},
		    ylabel = {$m$},
		    xmin=-1.5, xmax=9.5,
		    ymin=-3, ymax=5,
			axis x line=middle,
			axis y line=middle,
		    xticklabels = {0, , , , , $n - h_1$},
		    yticklabels = { ,$1 - n + h_2$, 0, $h_2 - h_1$, $n - h_1 - 1$}
		]
			\addplot coordinates {
				(1, 0)
				(8, 4)
				(8, 2)
				(1, -2)
				(1, 0)
			};
			\addplot[dashed, draw=gray, domain=0:8]{2};
			\addplot[dashed, draw=gray, domain=0:8]{4};
			\node at (axis cs:3.5, 2) [rotate=30, anchor=north east] {$m = t - 1$};
			\node at (axis cs:3.7, 0.2) [rotate=32, anchor=north east] {$m = t - n + h_2$};
		\end{axis}
	\end{tikzpicture}

	\caption{Замена переменных}
	\label{fig:label}
\end{figure}

Рассмотрим первый случай: $h_1 > h_2$. Поменяем порядок суммирования в \eqref{eq:cov_split}.
\begin{eqnarray*}
\nonumber
	& cov\{ 2 \tilde{\gamma}(h_1), 2 \tilde{\gamma}(h_2) \} = \\
	& = \frac{2}{(n - h_1) (n - h_2)} \sum_{t = 1}^{n - h_1}\sum_{t - m = 1}^{n - h_2} (\gamma(m - h_2) + \gamma(m + h_1) - \gamma(m) - \gamma(m + h_1 - h_2))^2 = \\
	& = \sum_{m = 1 - n + h_2}^{h_2 - h_1}\sum_{t = 1}^{m + n - h_2}(\gamma(m - h_2) + \gamma(m + h_1) - \gamma(m) - \gamma(m + h_1 - h_2))^2 + \\
	& + \sum_{m = h_2 - h_1 + 1}^{0}\sum_{t = 1}^{n - h_1}(\gamma(m - h_2) + \gamma(m + h_1) - \gamma(m) - \gamma(m + h_1 - h_2))^2 + \\
	& + \sum_{m = 1}^{n - h_1 - 1}\sum_{t = m + 1}^{n - h_1}(\gamma(m - h_2) + \gamma(m + h_1) - \gamma(m) - \gamma(m + h_1 - h_2))^2
\end{eqnarray*}

Заметим, что выражение под знаком суммы не зависит от $t$, получим:
\begin{eqnarray*}
\nonumber
	& cov\{ 2 \tilde{\gamma}(h_1), 2 \tilde{\gamma}(h_2) \} = \frac{2}{(n - h_1) (n - h_2)} \times \\
	& \times (\sum_{m = 1 - n + h_2}^{h_2 - h_1}(m + n - h_1)(\gamma(m - h_2) + \gamma(m + h_1) - \gamma(m) - \gamma(m + h_1 - h_2))^2 + \\
	& + (n - h_1)\sum_{m = h_2 - h_1 + 1}^{0}(\gamma(m - h_2) + \gamma(m + h_1) - \gamma(m) - \gamma(m + h_1 - h_2))^2 + \\
	& + \sum_{m = 1}^{n - h_1 - 1}(n - h_1 - m)(\gamma(m - h_2) + \gamma(m + h_1) - \gamma(m) - \gamma(m + h_1 - h_2))^2)
\end{eqnarray*}

Преобразуем полученное выражение:
\begin{eqnarray*}
\nonumber
	& cov\{ 2 \tilde{\gamma}(h_1), 2 \tilde{\gamma}(h_2) \} = \\
	& = \frac{2}{(n - h_1)(n - h_2)} ((n - h_1)\sum_{m = 1 - n + h_2}^{h_2 - h_1}(\gamma(m + h_1) + \gamma(m - h_2) - \gamma(m + h_1 - h_2) - \gamma(m))^2 + \\
	& + \sum_{m = 1 - n + h_2}^{h_2 - h_1} m (\gamma(m + h_1) + \gamma(m - h_2) - \gamma(m + h_1 - h_2) - \gamma(m))^2 \\
	& + (n - h_1)\sum_{m = h_2 - h_1 + 1}^{0} (\gamma(m + h_1) + \gamma(m - h_2) - \gamma(m + h_1 - h_2) - \gamma(m))^2 + \\
	& + (n - h_1)\sum_{m = 1}^{n - h_1 - 1} (\gamma(m + h_1) + \gamma(m - h_2) - \gamma(m + h_1 - h_2) - \gamma(m))^2 - \\
	& - \sum_{m = 1}^{n - h_1 - 1} m (\gamma(m + h_1) + \gamma(m - h_2) - \gamma(m + h_1 - h_2) - \gamma(m))^2)
\end{eqnarray*}

Приведем подобные:
\begin{eqnarray*}
\nonumber
	& cov\{ 2 \tilde{\gamma}(h_1), 2 \tilde{\gamma}(h_2) \} = \\
	& = \frac{2}{(n - h_2)} ( \sum_{m = 1 - n + h_2}^{n - h_1 - 1}(\gamma(m + h_1) + \gamma(m - h_2) - \gamma(m + h_1 - h_2) - \gamma(m))^2 + \\
	& + \frac{1}{(n - h_1)} ( \sum_{m = 1 - n + h_2}^{h_2 - h_1} m (\gamma(m + h_1) + \gamma(m - h_2) - \gamma(m + h_1 - h_2) - \gamma(m))^2 \\
	& - \sum_{m = 1}^{n - h_1 - 1} m (\gamma(m + h_1) + \gamma(m - h_2) - \gamma(m + h_1 - h_2) - \gamma(m))^2 ) )
\end{eqnarray*}

% section variogram_theory (end)