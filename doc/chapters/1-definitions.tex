%!TEX root = ../thesis.tex

\newpage
\chapter{Определения и вспомогательные результаты}
\label{c:definitions}

\section{Случайный процесс и его основные характеристики}

Для введения следующих понятий воспользуемся~\cite{brillinjer-ts, trush-ts}.

Пусть $ (\Omega, \mathcal{F}, P) $ --- вероятностное пространство, где $\Omega$ является произвольным множеством элементарных событий, $\mathcal{F}$ --- сигма-алгеброй подмножеств $\Omega$, и $P$ --- вероятностной мерой.

\begin{Definition}
\label{def:stochastic-process}
	\textit{Действительным случайным процессом} $ X(t) = X(\omega, t) $ называется семейство действительных случайных величин, заданных на вероятностном пространстве $ (\Omega, \mathcal{F}, P) $, где $ \omega \in \Omega, t \in \mathbb{T}$, где $ \mathbb{T} $ --- некоторое параметрическое множество.

	При $ \omega = \omega_{0} $, $ t \in \mathbb{T} $, $ X(\omega_{0}, t) $ является неслучайной функцией временного аргумента и называется \textit{траекторией случайного процесса}.

	При $ t = t_{0} $, $ \omega \in \Omega $, $ X(\omega, t_{0}) $ является случайной величиной и называется \textit{отсчетом случайного процесса}.
\end{Definition}

\begin{Definition}%!!!!!!!!!!!!!!!!!!!!!!!!!!!!!!!!!!!!!!!!!!!!!!!!!!!!!!!!!!!!
    Если $ \mathbb{T} = \{ 1, 2, \dots \}, (-\infty, \infty), [0, \infty), \dots $, то $ X(t), t \in \mathbb{T} $ называют \textit{случайным процессом с непрерывным временем}.
\end{Definition}

\begin{Definition}
	Если $ \mathbb{T} = \mathbb{Z} = \{ 0, \pm 1, \pm 2, \dots \} $, или $ \mathbb{T} \subset \mathbb{Z} $, то говорят, что $ X(t), t \in \mathbb{T} $, --- \textit{случайный процесс с дискретным временем}.
\end{Definition}

В дальнейшем в данной работе будем рассматривать случайные процессы с дискретным временем.

\begin{Definition}
\label{def:distr_func}
	\textit{n-мерной функцией распределения случайного процесса} $ X(t), t \in \mathbb{T} $, называется функция вида
	\begin{equation*}
		F_n(x_1, \dots, x_n; t_1, \dots, t_n) = P \{ X(t_1) < x_1, \dots, X(t_n) < x_n \},
	\end{equation*}
	где $ x_j \in \mathbb{R}, t_j \in \mathbb{T}, j = \overline{1,n} $.
\end{Definition}

\begin{Definition}
	\textit{Математическим ожиданием} случайного процесса $ X(t), t \in \mathbb{T}, $ называется функция вида
	\begin{equation*}
		m(t) = E \{ X(t) \} = \int \limits_{\mathbb{R}} x \, dF_1(x;t), t \in \mathbb{T}.
	\end{equation*}
\end{Definition}

\begin{Definition}
	\textit{Дисперсией} случайного процесса $ X(t), t \in \mathbb{T} $ называется функция вида:
	\begin{equation*}
		V(t) = {V \{ X(t) \} = E \{ X(t) - m(t) \}}^2 = \int \limits_{\mathbb{R}} {(x - m(t))}^2 \, dF_1(x; t).
	\end{equation*}
\end{Definition}

\begin{Definition}
	\textit{Ковариационной функцией} случайного процесса $ X(t), t \in \mathbb{T} $ называется функция вида:
	\begin{eqnarray*}
		& R(t_1, t_2) = cov\{ X(t_1), X(t_2) \} = E \{ (X(t_1) - m(t_1)) (X(t_2) - m(t_2)) \} = \\
		& = \iint \limits_{\mathbb{R}^2} (x_1 - m(t_1)) (x_2 - m(t_2)) \, dF_2(x_1, x_2; t_1, t_2)
	\end{eqnarray*}
\end{Definition}

\begin{Definition}
	\textit{Корреляционной функцией} случайного процесса $ X(t), t \in \mathbb{T} $, называется функция вида:
	\begin{equation*}
		r(t_1, t_2) = E \{ X(t_1)X(t_2) \} = \iint \limits_{\mathbb{R}^2} x_1 x_2 \, dF_2(x_1, x_2; t_1, t_2)
	\end{equation*}
\end{Definition}

\begin{Definition}
\label{def:corr_cov}
	\textit{Нормированной ковариационной функцией} называется функция вида
	\begin{equation*}
		corr\{ X(t_1), X(t_2)\} = \frac{cov\{ X(t_1), X(t_2) \}}{\sqrt{ V\{ X(t_1) \} V\{ X(t_2) \} }},
	\end{equation*}
	где $ X(t), t \in \mathbb{T} $, --- случайный процесс.
\end{Definition}

\begin{Definition}
    \textit{Вариограммой} случайного процесса $ X(t), t \in \mathbb{T} $, называется функция вида
	\begin{equation}
	    2 \gamma (h) = V \{ X(t + h) - X(t) \},~ t, h \in \mathbb{T}.
	\end{equation}

	При этом функция $ \gamma (h), h \in \mathbb{T} $, называется \textit{семивариограммой}. Нетрудно видеть, что $ \gamma(h) \ge 0 $, $ \gamma(0) = 0 $, $ \gamma(h) = \gamma(-h) $.
\end{Definition}

\section{Виды стационарности}
\label{sec:variogramAndInnerStationarity}
% Ссылка на литературу

\begin{Definition}
	Случайный процесс $ X(t), t \in \mathbb{T} $, называется \textit{стационарным в узком смысле}, если $ \forall n \in \mathbb{N} $, $ \forall t_1, \dots, t_n \in \mathbb{T} $, $ \forall \tau, t_1 + \tau, \dots, t_n + \tau \in \mathbb{T} $ выполняется соотношение:
	\begin{equation*}
		F_n(x_1, \dots, x_n; t_1, \dots, t_n) = F_n(x_1, \dots, x_n; t_1 + \tau , \dots, t_n + \tau).
	\end{equation*}
\end{Definition}

\begin{Definition}
\label{def:stat_wide}
	Случайный процесс $ X (t), t \in \mathbb{T} $, называется \textit{стационарным в широком смысле}, если $ \exists E \{ X^2(t) \} < \infty, t \in \mathbb{T} $, и
	\begin{enumerate}
		\item $ m(t) = E \{ X(t) \} = m = const, t \in \mathbb{T} $;
		\item\label{prop:covTimeInvariance} $ R(t_1, t_2) = R(t_1 - t_2), t_1,t_2 \in \mathbb{T} $.
	\end{enumerate}
\end{Definition}

\begin{Remark}
	Если случайный процесс $ X(t), t \in \mathbb{T} $, является стационарным в узком смысле и $ \exists E \{ X^2(t) \} < \infty, t \in \mathbb{T} $, то он будет стационарным и в широком смысле, но не наоборот.
\end{Remark}

\begin{Definition}
\label{eq:innerstat}
	Случайный процесс $ X(t),~ t \in \mathbb{T} $, называется \textit{внутренне стационарным}, если справедливы следующие равенства:
	\begin{equation}
	\label{eq:instat_expected}
		E \{ X(t_1) - X(t_2) \} = 0,
	\end{equation}
	\begin{equation}
	  V \{ X(t_1) - X(t_2) \} = 2 \gamma (t_1 - t_2),
	\end{equation}
	где $ 2 \gamma(t_1 - t_2) $ --- вариограмма рассматриваемого процесса, $ t_1, t_2 \in \mathbb{T} $.
\end{Definition}

\begin{Remark}
	Если $ X(t), t \in \mathbb{T} $, стационарный в широком смысле случайный процесс с ковариационной функцией $ R(t), t \in \mathbb{T} $, и семивариограммой $ \gamma(t), t \in \mathbb{T} $, то
	\begin{equation*}
		\gamma(t) = R(0) - R(t), \quad t \in \mathbb{T}.
	\end{equation*}
\end{Remark}

% Связь стац в широком смысле и внутр стац
\begin{Theorem}
	Пусть случайный процесс $ X(t), t \in \mathbb{T} $, является стационарным в широком смысле, тогда он будет и внутренне стационарным.
\end{Theorem}
\begin{proof}

По определению~\eqref{def:stat_wide} стационарного в широком смысле процесса
\begin{equation*}
	E \{ X(t_1) \} = m = const, \quad E \{ X(t_2) \} = m = const,
\end{equation*}
тогда, в силу линейности математического ожидания, выполняется свойство~\eqref{eq:instat_expected} определения внутренне стационарного процесса:
\begin{equation*}
	E \{ X(t_1) - X(t_2) \} = 0
\end{equation*}

Используя свойства дисперсии, запишем
\begin{equation*}\begin{gathered}
	V\{X(t_1) - X(t_2)\} = cov\{X(t_1) - X(t_2), X(t_1) - X(t_2)\} = \\
	= cov\{X(t_1), X(t_1)\} - 2cov\{X(t_1), X(t_2)\} + cov\{X(t_2), X(t_2)\}.
\end{gathered}\end{equation*}

Не нарушая общности считаем, что $ t_2 \le t_1 $. Используя свойство~\ref{prop:covTimeInvariance} стационарного в широком смысле процесса, получаем
\begin{equation*}\begin{gathered}
	V\{X(t_1) - X(t_2)\} = cov\{X(t_1 - t_2), X(t_1 - t_2)\} - \\
	- 2cov\{X(t_1 - t_2), X(0)\} + cov\{X(0), X(0)\} = \\
	= cov\{X(t_1 - t_2) - X(0), X(t_1 - t_2) - X(0)\} = V\{X(t_1 - t_2) - X(0)\}.
\end{gathered}\end{equation*}

Таким образом получена функция, зависящая только от $ t_1 - t_2 $, что и требовалось доказать.
\end{proof}

\begin{Remark}
	Обратное утверждение неверно. Пример процесса, который является внутренне стационарным, но не является стационарным в широком смысле --- винеровский процесс.
\end{Remark}
\begin{proof}
	Пусть $ X(t), t \in \mathbb{T} $ --- винеровский процесс, тогда по определению:
	\begin{enumerate}
		\item $ E \{ X(t) \} = 0 $;
		\item $ V \{ X(t_1) - X(t_2)\} = \sigma^{2}(t_1 - t_2) $.
	\end{enumerate}
	Следовательно процесс является внутренне стационарным.

  Однако по свойству винеровского процесса, $ cov \{ X(t_1 + r), X(t_1)\} = \min(t_1 + r, t_1) $. Не нарушая общности, считаем $ r > 0 $. Тогда $ cov \{ X(t_1 + r), X(t_1)\} = t_1 $ и $ cov \{ X(t_2 + r), X(t_2)\} = t_2 $. Так как в общем случае $ t_2 \ne t_1 $, то $ cov \{ X(t_1 + r), X(t_1)\} \ne cov \{ X(t_2 + r), X(t_2)\} $, а значит процесс не является стационарным в широком смысле.
\end{proof}

\begin{Theorem}
	Произвольная функция $ \gamma(t), t \in \mathbb{T} $, является семивариограммой некоторого внутренне стационарного случайного процесса $ X(t), t \in \mathbb{T} $ тогда и только тогда, когда $ \forall a > 0 $, функция $ e^{-a \gamma(t)}, t \in \mathbb{T}, $ является неотрицательно определённой.
\end{Theorem}
\begin{proof}
	Докажем необходиость. Пусть $ \gamma(t), t \in \mathbb{T}, $ --- семивариограмма внутренне стационарного случайного процесса $ X(t), t \in \mathbb{T} $. Нужно показать, что для $ \forall a > 0 $ функция $ e^{-a \gamma(t)}, t \in \mathbb{T}, $ является неотрицательно определённой.

	Заметим, что по свойствам семивариограммы, функция $ e^{-a \gamma(t)} $ является непрерывной, чётной и выпуклой книзу функцией, при этом истинны соотношения:
	\begin{enumerate}
		\item $ e^{-a \gamma(t)} \ge 0 $;
		\item $ e^{-a \gamma(0)} = 1 $;
		\item $ e^{-a \gamma(t)} \to 0 $ при $ t \to \infty $.
	\end{enumerate}
	Таким образом, функция $ e^{-a \gamma(t)} $ в полной мере удовлетворяет условиям теоремы Пойа~\cite[c. 303]{shiryaev1980} и, следовательно, является характеристической функцией.

	Тогда согласно теореме Бохнера-Хинчина~\cite[c. 303]{shiryaev1980} $ \forall a > 0 $ функция $ e^{-a \gamma(t)} $ является неотрицательно определённой.

	Обратно, пусть $ \forall a > 0, e^{-a \gamma(t)}, t \in \mathbb{T}, $ --- неотрицательно определённая функция. Нужно доказать, что функция $ \gamma(t) $ является семивариограммой внутренне стационарного случайного процесса.

	Согласно теореме Бохнера-Хинчина~\cite[c. 303]{shiryaev1980}, неотрицательно определённая функция $ e^{-a \gamma(t)} $ является характеристической функцией.

	Тогда, по теореме Пойа~\cite[c. 303]{shiryaev1980} $ e^{-a \gamma(t)}, \forall a > 0, $ является непрерывной, чётной и выпуклой книзу, причем выполняются:
	\begin{enumerate}
		\item $ e^{-a \gamma(t)} \ge 0 $;
		\item $ e^{-a \gamma(0)} = 1 $;
		\item $ e^{-a \gamma(t)} \to 0 $ при $ t \to \infty $.
	\end{enumerate}

	\ldots

	Нужно перейти к семивариограмме.

	\ldots

\end{proof}
