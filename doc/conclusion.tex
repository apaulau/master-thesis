% (PARTIAL)
\chapter*{Заключение}
\addcontentsline{toc}{chapter}{Заключение}

В представленной работе бы проведён сравнительный анализ современных пакетов прикладных программ для статистического анализа. Из них как инструмент исследования был выбран язык программирования \textbf{R}, по причине его доступности и предоставления огромного числа пакетов. С помощью этого пакета была исследована важнейшая характеристика любого водоёма --- температура воды. Исследование проводилось на основе данных, полученных из наблюдений за озером Баторино, в период с 1975 по 2012 год в июле месяце. Для этого были вычислены и проанализированы описательные статистики, проведена проверка на нормальность, проведён визуальный анализ. В результате указанной части работы было обнаружено, что распределение температуры воды в озере Баторино близко к номральному закону распределения с параметрами $\mathcal{N}(20.08, 5.24)$. Отклонение от нормальности отмечается полученными коэффициентами асимметрии и эксцесса. Исследуемое распределение имеет небольшую скошенность вправо и более растянутую колоколообразную форму относительно нормального закона распределения. В результате проведённого корреляционного анализа была выявлена умеренная зависимость между температурой воды и временем: был обнаружен рост температуры с течением времени.

В работе, как заключительный этап исследования, был проведён регрессионный анализ. В процесса которого была построена аддитивная модель временого ряда, найдён тренд, и, как следствие удаления тренда из построенной модели, был получен ряд остатков. Построенная детерминированными методами линейная регрессионная модель оказалась значимой и адекватной, но при этом описывает поведение временного ряда лишь частично. В результате анализа ряда остатков было выявлено отклонение распределения от нормальности. Что говорит о наличии некоторых неучтённых данной моделью факторов, затрудняющих дальнейшее исследование классическими методами. Следует также отметить стационарность и отсутствие автокорреляций в ряде остатков. Эти результаты говорят о постоянстве вероятностных свойств с течением времени, а также об отсутствии зависимостей между наблюдениями.

В заключение хотелось бы отметить, что представленные в данной работе классические методы анализа временных рядов, в этом случае оказались недостаточными для полноценного исследования. Поэтому в дальнейшем исследовании следует использовать отличные от использованных в работе современные методы анализа.