\newpage

\chapter*{Введение}
\addcontentsline{toc}{chapter}{Введение}

Данная работа посвящена статистическому анализу, обработке и исследованию реальных временных рядов. В настоящее время, выбор такой направленности соответствует необходимости в проведении анализа различных длительных наблюдений с математической и, в частности, статистической точек зрения. Поскольку наличие даже большого количества информации, полученной в процессе каких-либо наблюдений, не всегда способно раскрыть те или иные причины и следствия, представленные в конкретном случае. Наличие информации, без последующего всестороннего анализа, также не может раскрыть все скрытые проблемы и свойства. В свою очередь, математический аппарат и его конкретные прикладные части могут позволить не только проанализировать сложившуюся ситуацию, но и постараться дать некоторый прогноз по состоянию в будущем.

В качестве исследуемого материала в данной работе используются база данных с реальными наблюдениями, зафиксированными на озёрах, входящих в Нарочанский национальный парк, за период с 1955 по 2012 годы, полученная от учебно-научного центра <<Нарочанская биологическая станция им. Г.Г.Винберга>>. В представленной базе данных присутсвовали наблюдения за следующими озёрами:
\begin{itemize}
\item Баторино
\item Нарочь
\item Мястро
\end{itemize}
Из представленных озёр для исследования было выбрано озеро Баторино. Данное озеро является уникальным природным объектом, изучение которого позволяет решать проблемы экологии не только в региональном, но и глобальном масштабе. Оно располагается у самой границы города Мядель и входит в состав Нарочанской озёрной группы. Кроме представленных ранее озер, в эту группу также входят озеро Белое и озеро Бледное.

В данной работе исследуемым показателем для озера Баторино был избран показатель температуры воды. Температура воды принадлежит к числу наиболее важных и фундаментальных характеристик любого водоёма. Её изменение во времени является одним из важнейших факторов, отражающих изменения в окружающей среде. Также нужно отметить, что исследование температуры воды является актуальным, вследствие зависимостей свойств воды от температуры. Так как данная характеристика оказывает сильное влияние на плотности воды, растворимость в ней газов, минеральных и органических веществ, в том числе одни из важнейших характеристик для обитания в ней живых организмов: растворимость и насыщенность воды кислородом. В частности от температуры воды в значительной мере зависит жизнедеятельность рыб: их распределение в водоёме, питание, размножение. К тому же температура тела рыб, как правило, не превышает температуры окружающей их воды. В то же время, любой водоём как экосистема является средой обитания различных, отличных от рыб, организмов. И поэтому отслеживание всех изменений и влияние этих изменений на их жизнь является крайне важным не только в экологическом смысле, но и в биологическом. Как следствие вышесказанного, изменения температуры с течением времени следует считать одним из важнейших индикаторов изменений, происходящих в экосистеме озера. А исследование данного показателя, в свою очередь, является важнейшим в исследовании различных проблем, связанных с водоёмами. В подтверждение актуальности исследования представленной темы можно привести работы. \todo{Добавить ссылки}

В настоящее время, в условиях глобального потепления и крайне нестабильной климатической ситуации, наблюдения за состоянием озёрных экосистем представляют особую ценность как с научной, так и с практической стороны, поскольку только на основе таких наблюдений возможно выделить последствия антропогенного воздействия на фоне изменения природных факторов. А также получить некоторые заключения по экологической обстановке в определенной области.

Изначально, для решения этой задачи был выбран пакет \textbf{STATISTICA}. Данный программный пакет --- это универсальная интегрированная система, предназначенная для статистического анализа и визуализации данных, управления базами данных и разработки пользовательских приложений, содержащая широкий набор процедур анализа для применения в научных исследованиях, технике, бизнесе, а также специальные методы добычи данных.
В системе \textbf{STATISTICA} реализовано множество мощных языков программирования, которые снабжены специальными средствами поддержки. С их помощью легко создаются законченные пользовательские решения и встраиваются в различные другие приложения или вычислительные среды. 
В пакете представлены несколько сотен типов графиков 2D, 3D и 4D, матрицы и пиктограммы; предоставляется возможность разработки собственного дизайна графика. Средства управления графиками позволяют работать одновременно с несколькими графиками, изменять размеры сложных объектов, добавлять художественную перспективу и ряд специальных эффектов, разбивку страниц и быструю перерисовку. Например, 3D-графики можно вращать, накладывать друг на друга, сжимать или увеличивать.

Но пакет \textbf{STATISTICA}, являясь по сути узкоспециализированным пакетом статистического анализа, наравне с преимуществами имеет и свои недостатки. Главным из которых, на мой взгляд, являеется то, что \textbf{STATISTICA} --- коммерческий продукт. И как следствие имеет закрытую платформу, закрытый исходный код и при этом является достаточно дорогим. Также, к недостаткам я бы отнёс довольно громоздкий интерфейс, и взаимодействие с программой только на уровне кнопок и таблиц. Недостаток в гибкости и открытости в некоторых ситуациях ограничивает возможности анализа. С другой стороны, такой подход позволяет быстро получать различные результаты без длительного изучения самого продукта. Но, объективно оценив все преимущества и недостатки, следует отметить, что пакет \textbf{STATISTICA} не является оптимальным, во всестороннем смысле, выбором.

Существует множество различных программных пакетов, с помощью которых можно осуществлять статистический анализ. Приведём основные из них:
\begin{itemize}
\item Excel
\item STATISTICA
\item SAS
\item JMP
\item SPSS
\item SPlus
\item R
\item Mathematica
\item Matlab
\end{itemize}
Каждый из представленных программных пакетов имеет свои преимущества и недостатки. Следует сразу отметить, что такие пакеты как \textbf{Mathematica} и \textbf{Matlab}, которые являются по сути общематематическими, хоть и имеют статистические модули, но все же не являются специализорованным решением. Прежде всего, я хотел бы отметить преимущества пакета \textbf{STATISTICA} перед программой \textbf{Excel}. Во-первых, \textbf{STATISTICA} является специализированным пакетом по обработке статистических данных, вследствие чего этот процесс намного быстрее аналогичного в \textbf{Excel}. Во-вторых, в выбранном пакете присутствуют все необходимые для анализа данных инструменты и формулы, многое из которого нет в \textbf{Excel}. Если вкратце подвести итог, то \textbf{Excel} следует рассматривать скорее как иструмент работы с таблицами, а \textbf{STATISTICA} является платным и довольно громоздким пакетом, сходным по представлению данных (в таблицах) с \textbf{Excel}. Каждый из оставшихся в списке пакетов заслуживает отдельного ознакомления. Но если рассмотреть весь этот список в совокупности, то у каждого пакета, за исключением одного --- пакета \textbf{R} --- можно выделить одну общую черту: все они являются проприетарными и являются платными. И каждая новая версия продукта, при необходимости в оной, нуждается в приобретении. На самом деле, данную черту можно трактовать и как недостаток, и как преимущество. Следует отметить, что крупные коммерческие компании, заинтересованные в статистическом анализе, предпочитают такие пакеты как \textbf{SAS}, \textbf{JMP}, \textbf{SPSS}, \textbf{SPlus}. Поскольку покупая тот или иной продукт, компания получает гарантию в точности полученных с помощью пакета результатов, а также наличие оплаченой технической поддержке. Тогда как \textit{open-source} продукты предоставляются в комплекте ''как есть''. Но в совокупности с открытостью исходного кода и наличием огромных сообществ, отсутствие гарантии является слабым недостатком, так как у пользователя всегда есть возможность получить последнюю версию продукта, техническую помощь, или вовсе предложить свою помощь в разработке. Поэтому отдельные исследователи всё чаще и чаще выбирают пакет \textbf{R}.

Именно согласно этим соображениям, в качестве инструмента исследования в данной работе был выбран пакет \textbf{R}. Рассмотрим его подробнее. \textbf{R} является функциональным интерпретируемым языком программирования с динамической типизацией данных для статистической обработки данных и работы с графикой, а также свободная программная среда вычислений с открытым исходным кодом в рамках проекта \textit{GNU}. Язык создавался как аналогичный языку \textbf{S}, разработанному в \textit{Bell Labs} и является его альтернативной реализацией, хотя между языками есть существенные отличия, но в большинстве своём код на языке \textbf{S} работает в среде \textbf{R}.

\textbf{R} широко используется как статистическое программное обеспечение для анализа данных и фактически стал стандартом для статистических программ. Доступен для широкого числа операционных систем: \textit{Unix/Linux}, \textit{Microsoft Windows}, \textit{Mac OS X}.

\textbf{R} поддерживает широкий спектр статистических и численных методов и обладает хорошей расширяемостью с помощью пакетов. Пакеты представляют собой библиотеки для работы специфических функций или специальных областей применения. В базовую поставку R включен основной набор пакетов, а всего по состоянию на момент написания данной работы доступно более 5000 пакетов, которые распространяются через \textit{CRAN}(акроним \textit{Comprehensive R Archive Network}). С помощью них можно значительно расширить возможности для статистического анализа. Причем многие из них написаны самими пользователями. Как следствие этого, при отсутствии какого-либо функционала, всегда можно реализовать его посредством создания своего пакета или просто функции. Следует также отметить возможность использования напрямую функций, написанных на языках программирования \textit{C}, \textit{C++}, \textit{Java}.

Ещё одной особенностью \textbf{R} являются графические возможности, заключающиеся в возможности создания для публикаций качественной графики, которая может включать в себя математические символы. Динамические и интерактивные графики также доступны в качестве дополнительных пакетов.

\textbf{R} имеет свой собственный \LaTeX-подобный формат документации, который может быть использован для всестороннего документирования, как в режиме онлайн, в различных вариантах, так и в бумажном носителе.

Следует также отметить, что уже упомянутые ранее программные пакеты статистического анализа, на данный момент почти все имеют поддержку кода, написанного на \textbf{R}. Это о многом говорит, и уже не стоит удивлятся, если известный коммерческий продукт в очередном выпуске добавит аналогичную поддержку.

Два-три раза в год выходит свободно-распространяемый информационный журнал \textit{R Journal}. Он содержит информацию по статистической обработке данных и разработке, что может быть интересно как пользователям, так и разработчикам \textbf{R}.

Одна из самых популярных конференций, посвящённых языку --- \textit{useR! (The R User Conference)}, проходит ежегодно, начиная с 2004 года, собирает специалистов в различных областях.

Начиная с 2009 года каждой весной в Чикаго проводится конференция, посвящённая применению \textbf{R} в финансах (\textit{R/Finance: Applied Finance with R}). В 2013 году прошла первая конференция, посвящённая применению \textbf{R} в страховании (\textit{R in Insurance}).

\todo[inline, color=pink!40]{В Задание и Заключение включить пункт ``сделать/сделан сравнительный анализ прогр. пакетов статистической обработки данных''}