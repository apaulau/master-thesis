% (PARTIAL)
\newpage

\chapter*{Введение}
\addcontentsline{toc}{chapter}{Введение}

Данная работа посвящена статистическому анализу, обработке и исследованию реальных временных рядов. В настоящее время, выбор такой направленности соответствует необходимости в проведении анализа различных длительных наблюдений с математической и, в частности, статистической точек зрения. Поскольку наличие даже большого количества информации, полученной в процессе каких-либо наблюдений, не всегда способно раскрыть те или иные причины и следствия, представленные в конкретном случае. Наличие информации, без последующего всестороннего анализа, также не может раскрыть все скрытые проблемы и свойства. В свою очередь, математический аппарат и его конкретные прикладные части могут позволить не только проанализировать сложившуюся ситуацию, но и постараться дать некоторый прогноз по состоянию в будущем.

В качестве исследуемого материала в данной работе используются база данных с реальными наблюдениями, зафиксированными на озёрах, входящих в Нарочанский национальный парк, за период с 1955 по 2012 годы, полученная от учебно-научного центра <<Нарочанская биологическая станция им. Г.Г.Винберга>>. В представленной базе данных присутсвовали наблюдения за следующими озёрами Баторино, Нарочь, Мястро. Из них для исследования было выбрано озеро Баторино. Данное озеро является уникальным природным объектом, изучение которого позволяет решать проблемы экологии не только в региональном, но и глобальном масштабе. Оно располагается у самой границы города Мядель и входит в состав Нарочанской озёрной группы. Кроме представленных ранее озер, в эту группу также входят озеро Белое и озеро Бледное.

В данной работе исследуемым показателем озера Баторино было выбрано значение температуры воды. Температура воды принадлежит к числу наиболее важных и фундаментальных характеристик любого водоёма. Её изменение во времени является одним из важнейших факторов, отражающих изменения в окружающей среде. Также нужно отметить, что исследование температуры воды является актуальным, вследствие зависимостей свойств воды от температуры. Так как данная характеристика оказывает сильное влияние на плотности воды, растворимость в ней газов, минеральных и органических веществ, в том числе одни из важнейших характеристик для обитания в ней живых организмов: растворимость и насыщенность воды кислородом. В частности от температуры воды в значительной мере зависит жизнедеятельность рыб: их распределение в водоёме, питание, размножение. К тому же температура тела рыб, как правило, не превышает температуры окружающей их воды. В то же время, любой водоём как экосистема является средой обитания различных, отличных от рыб, организмов. И поэтому отслеживание всех изменений и влияние этих изменений на их жизнь является крайне важным не только в экологическом смысле, но и в биологическом. Как следствие вышесказанного, изменения температуры с течением времени следует считать одним из важнейших индикаторов изменений, происходящих в экосистеме озера. А исследование данного показателя, в свою очередь, является важнейшим в исследовании различных проблем, связанных с водоёмами. В подтверждение актуальности исследования представленной темы можно привести научные работы \cite{Katz2011,OBrien2012a,Subehi2011} аналогичных выбранной теме направлений.

Среди представленных следует отметить научную работу \cite{Katz2011}. Представленная статья рассматривает в качестве объекта исследования крупнейшее в мире озеро --- Байкал. В ней подробно изучается изменение климата в контексте данного озера в период с 1950 по 2012.

В работе \cite{OBrien2012a} исследуется температура воды Великих озёр в Северной Америке. А также исследуется влияние, оказываемое изменением температуры на рыб, обитающих в этих озёрах.

Следует отметить, что указанные работы опубликованы в 2011-2012, что также подтверждает актуальность темы исследований.

В настоящее время, в условиях глобального потепления и крайне нестабильной климатической ситуации, наблюдения за состоянием озёрных экосистем представляют особую ценность как с научной, так и с практической стороны, поскольку только на основе таких наблюдений возможно выделить последствия антропогенного воздействия на фоне изменения природных факторов. А также получить некоторые заключения по экологической обстановке в определенной области.

Изначально, для решения этой задачи был выбран пакет \textbf{STATISTICA}. Но данный пакет, являясь по сути узкоспециализированным пакетом статистического анализа, имеет недостаток в гибкости и открытости, вследствие чего, в некоторых случаях ограничивает возможности анализа. Именно согласно этим соображениям, в качестве инструмента исследования в данной работе был выбран пакет \textbf{R}. Поскольку \textbf{R} является функциональным интерпретируемым языком программирования с динамической типизацией данных для статистической обработки данных и работы с графикой, а также свободная программная среда вычислений с открытым исходным кодом.

%\todo[inline, color=pink!40]{В Задание и Заключение включить пункт ``сделать/сделан сравнительный анализ прогр. пакетов статистической обработки данных''}