%!TEX root = coursework.tex
\newpage

\makeatletter
\renewcommand\appendixname{Приложение}
\gdef\appendix{\par
\setcounter{section}{0}%
\setcounter{subsection}{0}%
\def\thesection{\@Alph\c@section}
\def\section##1{\refstepcounter{section}
\setcounter{table}{0}
\def\thetable{\@Alph\c@section.\arabic{table}}
\goodbreak\noindent{\indent\Large\bf\hbox to 8.5em{\appendixname\
\thesection\hss}\hspace{0.8ex}##1}%
\nobreak\vspace{1em}\nobreak\noindent%

\addcontentsline{toc}{chapter}{\hbox to 8.5em{\appendixname\ \thesection\hss}\hspace{0.8ex}##1}}}
\makeatother

\appendix

\section{ Исходные данные}
\label{c:source_data}
% latex table generated in R 3.3.0 by xtable 1.8-2 package
% Thu Jun 16 15:29:58 2016
\begin{table}[H]
\centering
\caption{Исходные данные.} 
\label{table:source}
\begin{tabular}{|rc|}
  \hline
Год & Температура, С \\ 
  \hline
1975 & 20.20 \\ 
  1976 & 16.00 \\ 
  1977 & 17.70 \\ 
  1978 & 16.75 \\ 
  1979 & 17.50 \\ 
  1980 & 16.77 \\ 
  1981 & 19.80 \\ 
  1982 & 19.00 \\ 
  1983 & 21.40 \\ 
  1984 & 19.40 \\ 
  1985 & 20.40 \\ 
  1986 & 16.50 \\ 
  1987 & 17.10 \\ 
  1988 & 23.80 \\ 
  1989 & 19.90 \\ 
  1990 & 18.50 \\ 
  1991 & 23.00 \\ 
  1992 & 21.90 \\ 
  1993 & 18.00 \\ 
  1994 & 21.40 \\ 
  1995 & 18.90 \\ 
  1996 & 19.10 \\ 
  1997 & 21.00 \\ 
  1998 & 18.40 \\ 
  1999 & 23.50 \\ 
  2000 & 21.00 \\ 
  2001 & 24.20 \\ 
  2002 & 23.10 \\ 
  2003 & 18.00 \\ 
  2004 & 19.10 \\ 
  2005 & 20.00 \\ 
  2006 & 21.30 \\ 
  2007 & 19.40 \\ 
  2008 & 21.80 \\ 
  2009 & 21.90 \\ 
  2010 & 24.30 \\ 
  2011 & 22.80 \\ 
  2012 & 20.20 \\ 
   \hline
\end{tabular}
\end{table}


\newpage
\section{ Графические материалы}
\label{c:graphs}

\setcounter{figure}{0}

\begin{figure}[H]
	\center{\includegraphics[width=1\linewidth]{../figures/residual/time-series.png}}
\caption{График ряда остатков}
\label{img:ts_detrended}
\end{figure}

\begin{figure}[H]
	\center{\includegraphics[width=1\linewidth]{../figures/residual/histogram.png}}
\caption{Гистограмма остатков с кривой плотности нормального распределения $\mathcal{N}(19.88, 4.92)$}
\label{img:resid_hist}
\end{figure}

\begin{figure}[H]
	\center{\includegraphics[width=1\linewidth]{../figures/variogram/manual-model.png}}
\caption{Экспериментальная и теоретическая вариограмма (сферическая модель)}
\label{img:manual-mod}
\end{figure}

\begin{figure}[H]
	\center{\includegraphics[width=1\linewidth]{../figures/variogram/classical-empirical.png}}
\caption{Экспериментальная вариограмма (классическая оценка)}
\label{img:classical_emp}
\end{figure}

\begin{figure}[H]
	\center{\includegraphics[width=1\linewidth]{../figures/variogram/manual-model.png}}
\caption{Экспериментальная и теоретическая вариограмма (классическая оценка)}
\label{img:classical-mod}
\end{figure}

\begin{figure}[H]
	\center{\includegraphics[width=1\linewidth]{../figures/variogram/robust-modeled.png}}
\caption{Экспериментальная и теоретическая вариограмма (робастная оценка)}
\label{img:robust-mod}
\end{figure}

\begin{figure}[H]
\center{\includegraphics[width=1\linewidth]{../figures/variogram/cross-prediction-robust-best.png}}
\caption{Сравнение прогнозных значений}
\label{img:cross-prediction-best}
\end{figure}

\newpage
\section{ Результаты вычислений}
\label{c:app_results}

% latex table generated in R 3.3.0 by xtable 1.8-2 package
% Thu Jun 16 15:30:38 2016
\begin{table}[H]
\centering
\caption{Временной ряд остатков} 
\label{table:residuals}
\begin{tabular}{|rc|}
  \hline
Год & Температура, С \\ 
  \hline
1975 & 2.13 \\ 
  1976 & -2.18 \\ 
  1977 & -0.59 \\ 
  1978 & -1.65 \\ 
  1979 & -1.01 \\ 
  1980 & -1.84 \\ 
  1981 & 1.07 \\ 
  1982 & 0.16 \\ 
  1983 & 2.45 \\ 
  1984 & 0.34 \\ 
  1985 & 1.23 \\ 
  1986 & -2.78 \\ 
  1987 & -2.29 \\ 
  1988 & 4.30 \\ 
  1989 & 0.29 \\ 
  1990 & -1.21 \\ 
  1991 & 3.18 \\ 
  1992 & 1.97 \\ 
  1993 & -2.04 \\ 
  1994 & 1.25 \\ 
  1995 & -1.36 \\ 
  1996 & -1.27 \\ 
  1997 & 0.52 \\ 
  1998 & -2.19 \\ 
  1999 & 2.80 \\ 
  2000 & 0.19 \\ 
  2001 & 3.28 \\ 
  2002 & 2.07 \\ 
  2003 & -3.14 \\ 
  2004 & -2.15 \\ 
  2005 & -1.36 \\ 
  2006 & -0.17 \\ 
   \hline
\end{tabular}
\end{table}

\input{../out/variogram/prediction-manual.tex}

\newpage
\section{ Код программ}
\label{c:listings}
\renewcommand{\thelstlisting}{D.1}
\lstinputlisting[language=R, caption=Описательные статистики, label=lst:dstats]{../R/lib/dstats.R}
\renewcommand{\thelstlisting}{D.2}
\lstinputlisting[language=R, caption=Основной код программы, label=lst:main]{../R/master.R}
\renewcommand{\thelstlisting}{D.3}
\lstinputlisting[language=R, caption=Вариограммный анализ, label=lst:variogram]{../R/predictor.R}