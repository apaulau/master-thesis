%!TEX root = ../thesis.tex

\newpage

\appendix

\section{ Исходные данные}
\label{c:source_data}
% latex table generated in R 3.3.0 by xtable 1.8-2 package
% Thu Jun 16 15:29:58 2016
\begin{table}[H]
\centering
\caption{Исходные данные.} 
\label{table:source}
\begin{tabular}{|rc|}
  \hline
Год & Температура, С \\ 
  \hline
1975 & 20.20 \\ 
  1976 & 16.00 \\ 
  1977 & 17.70 \\ 
  1978 & 16.75 \\ 
  1979 & 17.50 \\ 
  1980 & 16.77 \\ 
  1981 & 19.80 \\ 
  1982 & 19.00 \\ 
  1983 & 21.40 \\ 
  1984 & 19.40 \\ 
  1985 & 20.40 \\ 
  1986 & 16.50 \\ 
  1987 & 17.10 \\ 
  1988 & 23.80 \\ 
  1989 & 19.90 \\ 
  1990 & 18.50 \\ 
  1991 & 23.00 \\ 
  1992 & 21.90 \\ 
  1993 & 18.00 \\ 
  1994 & 21.40 \\ 
  1995 & 18.90 \\ 
  1996 & 19.10 \\ 
  1997 & 21.00 \\ 
  1998 & 18.40 \\ 
  1999 & 23.50 \\ 
  2000 & 21.00 \\ 
  2001 & 24.20 \\ 
  2002 & 23.10 \\ 
  2003 & 18.00 \\ 
  2004 & 19.10 \\ 
  2005 & 20.00 \\ 
  2006 & 21.30 \\ 
  2007 & 19.40 \\ 
  2008 & 21.80 \\ 
  2009 & 21.90 \\ 
  2010 & 24.30 \\ 
  2011 & 22.80 \\ 
  2012 & 20.20 \\ 
   \hline
\end{tabular}
\end{table}


\newpage
\section{ Графические материалы}
\label{c:graphs}

\setcounter{figure}{0}

\begin{figure}[H]
	\center{\includegraphics[width=1\linewidth]{../figures/residual/histogram.png}}
\caption{Гистограмма остатков с кривой плотности нормального распределения $\mathcal{N}(19.88, 4.92)$}
\label{img:resid_hist}
\end{figure}

\begin{figure}[H]
	\center{\includegraphics[width=1\linewidth]{../figures/residual/hscat.pdf}}
\caption{Диаграмма взаимного разброса}
\label{img:hscat}
\end{figure}

\begin{figure}[H]
	\center{\includegraphics[width=1\linewidth]{../figures/variogram/lin-modeled.png}}
\caption{Экспериментальная и теоретическая вариограмма $ 4 \cdot Lin(h, 0) $}
\label{img:lin-modeled}
\end{figure}

\begin{figure}[H]
	\center{\includegraphics[width=1\linewidth]{../figures/variogram/lin-fit-modeled.png}}
\caption{Экспериментальная и теоретическая вариограмма $ 4.08 \cdot Nug(h) $}
\label{img:lin-fit}
\end{figure}

\begin{figure}[H]
	\center{\includegraphics[width=1\linewidth]{../figures/variogram/lin-fit-cv-modeled.png}}
\caption{Экспериментальная и теоретическая вариограмма $ 4 \cdot Lin(h, 4) $}
\label{img:lin-fit-cv}
\end{figure}

\begin{figure}[H]
	\center{\includegraphics[width=1\linewidth]{../figures/variogram/lin-fit-cv-cross-prediction.png}}
\caption{Прогноз $ 4 \cdot Lin(h, 4) $}
\label{img:lin-fit-cv-pred}
\end{figure}

\begin{figure}[ht]
	\center{\includegraphics[width=1\linewidth]{../figures/static/lin-range-adapt.png}}
\caption{Зависимость качества линейной модели от значения ранга}
\label{img:lin-range-adapt}
\end{figure}

\begin{figure}[ht]
	\center{\includegraphics[width=1\linewidth]{../figures/variogram/lin-fit-adapt-modeled.png}}
\caption{Экспериментальная и теоретическая вариограмма $ 2 \cdot Lin(h, 2) $}
\label{img:lin-adapt-modeled}
\end{figure}

\begin{figure}[H]
	\center{\includegraphics[width=1\linewidth]{../figures/variogram/sph-fit-adapt-modeled.png}}
\caption{Экспериментальная и теоретическая вариограмма $ 0.9 + 4 \cdot Sph(h, 6.9) $}
\label{img:sph-adapt-modeled}
\end{figure}

\begin{figure}[H]
	\center{\includegraphics[width=1\linewidth]{../figures/variogram/sph-fit-adapt-cross-prediction.png}}
\caption{Прогноз $ 0.9 + 4 \cdot Sph(h, 6.9) $}
\label{img:sph-adapt-pred}
\end{figure}

\begin{figure}[H]
	\center{\includegraphics[width=1\linewidth]{../figures/variogram/per-fit-cv-modeled.png}}
\caption{Экспериментальная и теоретическая вариограмма $ 4 \cdot Per(h, 0.898) $}
\label{img:per-cv-modeled}
\end{figure}

\begin{figure}[H]
\center{\includegraphics[width=1\linewidth]{../figures/variogram/robust-variogram.png}}
\caption{Экспериментальная вариограмма (оценка Кресси-Хокинса)}
\label{img:robust-variogram}
\end{figure}

\begin{figure}[H]
\center{\includegraphics[width=1\linewidth]{../figures/variogram/auto-class-20-modeled.png}}
\caption{Экспериментальная и теоретическая вариограмма $ 1.011 \cdot Wav(h, 1.14) $}
\label{img:auto-class-modeled}
\end{figure}

\begin{figure}[H]
\center{\includegraphics[width=1\linewidth]{../figures/variogram/auto-class-20-cross-prediction.png}}
\caption{Прогноз $ 1.011 \cdot Wav(h, 1.14) $}
\label{img:auto-class-20-pred}
\end{figure}

\begin{figure}[H]
\center{\includegraphics[width=1\linewidth]{../figures/variogram/auto-class-26-cross-prediction.png}}
\caption{Прогноз $ 3.46 + 0.5 \cdot Per(h, 2.67) $}
\label{img:auto-class-26-pred}
\end{figure}

\begin{figure}[H]
\center{\includegraphics[width=1\linewidth]{../figures/variogram/auto-rob-5-cross-prediction.png}}
\caption{Прогноз $ 4.11 + 1.65 \cdot Wav(h, 3.59) $}
\label{img:auto-rob-5-pred}
\end{figure}

\begin{figure}[H]
\center{\includegraphics[width=1\linewidth]{../figures/variogram/auto-class-18-cross-prediction.png}}
\caption{Прогноз $ 3.8 + 0.32 \cdot Per(h, 1.3) $}
\label{img:auto-class-18-pred}
\end{figure}

\newpage
\section{ Результаты вычислений}
\label{c:app_results}

% latex table generated in R 3.3.0 by xtable 1.8-2 package
% Thu Jun 16 15:30:38 2016
\begin{table}[H]
\centering
\caption{Временной ряд остатков} 
\label{table:residuals}
\begin{tabular}{|rc|}
  \hline
Год & Температура, С \\ 
  \hline
1975 & 2.13 \\ 
  1976 & -2.18 \\ 
  1977 & -0.59 \\ 
  1978 & -1.65 \\ 
  1979 & -1.01 \\ 
  1980 & -1.84 \\ 
  1981 & 1.07 \\ 
  1982 & 0.16 \\ 
  1983 & 2.45 \\ 
  1984 & 0.34 \\ 
  1985 & 1.23 \\ 
  1986 & -2.78 \\ 
  1987 & -2.29 \\ 
  1988 & 4.30 \\ 
  1989 & 0.29 \\ 
  1990 & -1.21 \\ 
  1991 & 3.18 \\ 
  1992 & 1.97 \\ 
  1993 & -2.04 \\ 
  1994 & 1.25 \\ 
  1995 & -1.36 \\ 
  1996 & -1.27 \\ 
  1997 & 0.52 \\ 
  1998 & -2.19 \\ 
  1999 & 2.80 \\ 
  2000 & 0.19 \\ 
  2001 & 3.28 \\ 
  2002 & 2.07 \\ 
  2003 & -3.14 \\ 
  2004 & -2.15 \\ 
  2005 & -1.36 \\ 
  2006 & -0.17 \\ 
   \hline
\end{tabular}
\end{table}

% latex table generated in R 3.3.0 by xtable 1.8-2 package
% Tue Jun 21 13:42:54 2016
\begin{table}[H]
\centering
\caption{Прогнозные значения (модель $ \widehat{\gamma}_6(h) $)} 
\label{table:per-fit-cv-prediction}
\begin{tabular}{r|cccc}
  \hline
 & $X(t)$ & $X^{*}(t)$ & $y(t)$ & $ X(t) - X^{*}(t) $ \\ 
  \hline
2007 & 19.400 & 22.133 & 21.578 & -2.733 \\ 
  2008 & 21.800 & 23.106 & 21.687 & -1.306 \\ 
  2009 & 21.900 & 23.441 & 21.797 & -1.541 \\ 
  2010 & 24.300 & 23.028 & 21.906 & 1.272 \\ 
  2011 & 22.800 & 22.122 & 22.016 & 0.678 \\ 
  2012 & 20.200 & 21.219 & 22.126 & -1.019 \\ 
   \hline
\end{tabular}
\end{table}

% latex table generated in R 3.3.0 by xtable 1.8-2 package
% Tue Jun 21 13:44:52 2016
\begin{table}[H]
\centering
\caption{Прогнозные значения (модель $ \widehat{\gamma}_9(h) $)} 
\label{table:auto-rob-5-prediction}
\begin{tabular}{r|cccc}
  \hline
 & $X(t)$ & $X^{*}(t)$ & $y(t)$ & $ X(t) - X^{*}(t) $ \\ 
  \hline
2007 & 19.400 & 21.385 & 21.578 & -1.985 \\ 
  2008 & 21.800 & 21.877 & 21.687 & -0.077 \\ 
  2009 & 21.900 & 22.162 & 21.797 & -0.262 \\ 
  2010 & 24.300 & 22.217 & 21.906 & 2.083 \\ 
  2011 & 22.800 & 22.134 & 22.016 & 0.666 \\ 
  2012 & 20.200 & 22.054 & 22.126 & -1.854 \\ 
   \hline
\end{tabular}
\end{table}


\newpage
\section{ Код программ}
\label{c:listings}
\renewcommand{\thelstlisting}{Г.1}
\lstinputlisting[language=R, caption=Описательные статистики, label=lst:dstats]{../R/lib/dstats.R}
% \renewcommand{\thelstlisting}{Г.2}
% \lstinputlisting[language=R, caption=Основной код программы, label=lst:main]{../R/master.R}
\renewcommand{\thelstlisting}{Г.2}
\lstinputlisting[language=R, caption=Вариограммный анализ, label=lst:variogram]{../R/predictor.R}
\renewcommand{\thelstlisting}{Г.3}
\lstinputlisting[language=R, caption=Автоматический подбор моделей, label=lst:afv]{../R/lib/afv.R}