\documentclass[a4paper,12pt]{report}
\usepackage{cmap} % включает таблицу символов в .pdf --- кириллический текст можно искать
\usepackage{bsustyle/style/bsumain}
\usepackage{bsustyle/style/bsutitle}
\usepackage{bsustyle/style/lstcustom}
\usepackage[colorinlistoftodos]{todonotes}
\usepackage{listings}
\usepackage{appendix}

\definecolor{mauve}{rgb}{224, 176, 255}

\lstset{
	language=R,										% выбор языка для подсветки (здесь это R)
	basicstyle=\scriptsize\ttfamily,				% размер и начертание шрифта для подсветки кода
	commentstyle=\ttfamily\color{gray},				
	numbers=left,									% где поставить нумерацию строк (слева\справа)
	numberstyle=\tiny\color{gray}, % стиль шрифта для номеров строк
	stepnumber=1,									% размер шага между двумя номерами строк
	numbersep=5pt,									% как далеко отстоят номера строк от подсвечиваемого кода
	backgroundcolor=\color{white},					% цвет фона подсветки - используем \usepackage{color}
	showspaces=false,								% показывать или нет пробелы специальными отступами
	showstringspaces=false,							% показывать или нет пробелы в строках
	showtabs=false,									% показывать или нет табуляцию в строках
	frame=single,									% рисовать рамку вокруг кода
	rulecolor=\color{black},        
	tabsize=2,										% размер табуляции по умолчанию равен 2 пробелам
	captionpos=b,									% позиция заголовка вверху [t] или внизу [b] 
	breaklines=true,								% автоматически переносить строки (да\нет)
	breakatwhitespace=false,						% переносить строки только если есть пробел
	title=\lstname,
 	keywordstyle={},      
  	commentstyle={},   
  	stringstyle=\ttfamily,      
  	escapeinside={\%*}{*)},         
	morekeywords={acf,ar,arima,arima.sim,colMeans,colSums,is.na,is.null,%
    	mapply,ms,na.rm,nlmin,replicate,row.names,rowMeans,rowSums,seasonal,%
    	sys.time,system.time,ts.plot,which.max,which.min}
}

% todonotes settings.
\setlength{\marginparwidth}{2.7cm}
\reversemarginpar