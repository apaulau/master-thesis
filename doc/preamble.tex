\documentclass[a4paper,12pt]{report}
\usepackage{cmap} % включает таблицу символов в .pdf --- кириллический текст можно искать
\usepackage{bsustyle/style/bsumain}
\usepackage{bsustyle/style/bsutitle}
\usepackage{bsustyle/style/lstcustom}
\usepackage[colorinlistoftodos]{todonotes}
\newcommand{\unsure}[2][1=]{\todo[linecolor=red,backgroundcolor=red!25,bordercolor=red,#1]{#2}}
\newcommand{\change}[2][1=]{\todo[linecolor=blue,backgroundcolor=blue!25,bordercolor=blue,#1]{#2}}
\newcommand{\info}[2][1=]{\todo[linecolor=OliveGreen,backgroundcolor=OliveGreen!25,bordercolor=OliveGreen,#1]{#2}}
\newcommand{\improvement}[2][1=]{\todo[linecolor=Plum,backgroundcolor=Plum!25,bordercolor=Plum,#1]{#2}}
\newcommand{\thiswillnotshow}[2][1=]{\todo[disable,#1]{#2}}
% \usepackage{xcolor}
\usepackage{listings}
% \usepackage{appendix}

%\DeclareUnicodeCharacter{00A0}{ }

% КОСТЫЛЬ
% \makeatletter
% \newcounter{sect}
% \renewcommand{\thesect}{\Asbuk{sect}.\,}
% \newcommand{\sect}[1]{\refstepcounter{sect}\par\vspace{1.5cm plus 1cm minus .5cm}
% 	{\large\bfПРИЛОЖЕНИЕ~\thesect #1}%
% 	\addcontentsline{toc}{section}{ПРИЛОЖЕНИЕ~\thesect #1}\markboth{\thesect #1}{\thesect #1}%
% 	\nopagebreak\bigskip\par}
% \makeatother



% \makeatletter
% \renewcommand{\appendix}{\par%
%   \setcounter{section}{0}%
%   \setcounter{subsection}{0}% 
%   \renewcommand{\appendixname}{ПРИЛОЖЕНИЕ}%
%   \def\sectionname{\appendixname}%
%   \addtocontents{toc}{section}%
%   \renewcommand{\thesection}{\appendixname\hspace{0.2cm}\Asbuk\c@section}%
% }
% \makeatother

% \newcommand\myappendix[1]{

% \refstepcounter{section}

% \chapter*{Приложение~\thesection{}~#1}

% \addcontentsline{toc}{chapter}{Приложение~\thesection{}~#1}}

% \makeatletter 
% \renewcommand\appendix{\par 
% \setcounter{section}{0}% 
% \setcounter{subsection}{0}% 
% \gdef\thesection{ \@Asbuk\c@section}} 
% \makeatother 

\definecolor{mauve}{rgb}{224, 176, 255}

\lstset{
	language=R,										% выбор языка для подсветки (здесь это R)
	basicstyle=\scriptsize\ttfamily,				% размер и начертание шрифта для подсветки кода
	commentstyle=\ttfamily\color{gray},				
	numbers=left,									% где поставить нумерацию строк (слева\справа)
	numberstyle=\tiny\color{gray}, % стиль шрифта для номеров строк
	stepnumber=1,									% размер шага между двумя номерами строк
	numbersep=5pt,									% как далеко отстоят номера строк от подсвечиваемого кода
	backgroundcolor=\color{white},					% цвет фона подсветки - используем \usepackage{color}
	showspaces=false,								% показывать или нет пробелы специальными отступами
	showstringspaces=false,							% показывать или нет пробелы в строках
	showtabs=false,									% показывать или нет табуляцию в строках
	frame=single,									% рисовать рамку вокруг кода
	rulecolor=\color{black},        
	tabsize=2,										% размер табуляции по умолчанию равен 2 пробелам
	captionpos=b,									% позиция заголовка вверху [t] или внизу [b] 
	breaklines=true,								% автоматически переносить строки (да\нет)
	breakatwhitespace=false,						% переносить строки только если есть пробел
	title=\lstname,
 	keywordstyle={},      
  	commentstyle={},   
  	stringstyle=\ttfamily,      
  	escapeinside={\%*}{*)},         
	morekeywords={acf,ar,arima,arima.sim,colMeans,colSums,is.na,is.null,%
    	mapply,ms,na.rm,nlmin,replicate,row.names,rowMeans,rowSums,seasonal,%
    	sys.time,system.time,ts.plot,which.max,which.min}
}

% todonotes settings.
%\setlength{\marginparwidth}{2.7cm}
%\reversemarginpar