\documentclass[a4paper]{report}

\usepackage[T1,T2A]{fontenc}
\usepackage[utf8]{inputenc}
\usepackage[russian]{babel}
\usepackage{amssymb,amsthm,amsmath,amscd}
\usepackage{graphicx}

\usepackage[colorinlistoftodos]{todonotes}

%Оформление глав, разделов и т.д.
\makeatletter%

%не подавлять абзацный отступ в главах
\renewcommand{\chapter}{\cleardoublepage\thispagestyle{plain}%
\global\@topnum=0 \@afterindenttrue \secdef\@chapter\@schapter}

%оформление нумерованных глав
\renewcommand{\@makechapterhead}[1]{%Начало макроопределения
	\vspace*{50pt}%Пустое место вверху страницы
	{\parindent=18pt \normalfont\Large\bfseries
	\thechapter{} %номер главы
	\normalfont\Large\bfseries #1 \par %заголовок от текста
	\nopagebreak %чтоб не оторвать заголовок от текста
	\vspace{40 pt} %между заголовком и текстом
	}%конец группы
}%коней макроопределения

%оформление ненумерованных глав
\renewcommand{\@makeschapterhead}[1]{%Начало макроопределения
	\vspace*{50pt}%Пустое место вверху страницы
	{\parindent=18pt \normalfont\Large\bfseries #1 \par %заголовок от текста
	\nopagebreak %чтоб не оторвать заголовок от текста
	\vspace{40pt} %между заголовком и текстом
	}%конец группы
}%конец макроопределения

%оформление разделов
\renewcommand{\section}{\@startsection{section}{1}{18pt}%
{3.5ex plus 1ex minus .2ex}{2.3ex plus .2ex}%
{\normalfont\Large\bfseries\raggedright}}%

%оформление подразделов
\renewcommand{\subsection}{\@startsection{subsection}{2}{18pt}%
{3.25ex plus 1ex minus .2ex}{1.5ex plus .2ex}%
{\normalfont\large\bfseries\raggedright}}%

%оформление подподразделов
\renewcommand{\subsubsection}{\@startsection{subsubsection}{3}{18pt}%
{3.25ex plus 1ex minus .2ex}{1.5ex plus .2ex}%
{\normalfont\large\bfseries\raggedright}}%

%Определение полей
\setlength{\oddsidemargin}{0cm}% 1in=2.54см
\setlength{\hoffset}{0.46cm}% 1in+\hoffset=3cm = левое поле;

\setlength{\textwidth}{17cm}% 21cm-3cm(левое поле)-1cm(правое поле)=17cm;

\setlength{\headheight}{0cm}%
\setlength{\topmargin}{0cm}%
\setlength{\headsep}{0cm}%
\setlength{\voffset}{-0.54cm}% 1in+\voffset=2cm = верхнее поле;

\setlength{\textheight}{25.7cm}% 29.7cm-2cm(верхнее поле)-2cm(нижнее поле)=25.7cm;

\title{Philips HD2708/03}
\author{solis}
\begin{document}

\maketitle

\chapter*{Замечания}

Отключение режима, не выдергивая шнура: необходимо 2 раза нажать на кнопку "ВЫКЛ/Подогрев".

Отсчет начинается после создания давления в режиме "работа под давлением".

 В рецептах указывается время, которое уходит на само приготовление, но мультиварка-скороварка сначала нагревает емкость (горит индикатор «Нагрев») и время нагрева зависит от количества продукта, потом наступает режим создания давления ( горит индикатор «Создание давления») и только потом загорается индикатор «Работа под давлением», означающий, что приготовление началось и включается обратный отсчет времени. По завершении приготовления подается звуковой сигнал, автоматически включается режим «Подогрев» и начинается прямой отсчет времени подогрева. Кнопкой «Выкл/Подогрев» можно его выключить. Мультиварка-скороварка перейдет в режим ожидания.

Мультиварка-скороварка сначала нагревает емкость ( горит индикатор «Нагрев»)- и время нагрева зависит от количества продукта, потом наступает режим создания давления ( горит индикатор «Создание давления»). Затем загорается индикатор «Работа под давлением», означающий, что приготовление началось и включается обратный отсчет времени. По завершении приготовления подается звуковой сигнал, автоматически включается режим «Подогрев» и начинается прямой отсчет времени подогрева. Кнопкой «Выкл/Подогрев» можно его выключить. Мультиварка-скороварка перейдет в режим ожидания.


Пробуем готовить готовить так: если есть жидкость в продукте (включая выделяемый сок) --- клапан закройте; если сухой продукт (кекс) --- клапан откройте. Клапан закрываем, когда включаем режим. Открываем после сигнала о готовности и сбросе избыточного давления. Попробуйте на простых продуктах так готовить (жидкость-закрыт, нет жидкости - открыт). Только время не забудьте сократить, для кусочков курицы достаточно 20 минут на режиме, например.

Что касается мяса, можно применять рецепты из прилагаемых в комплекте. По мясу рецепты адекватные. Очень часто тушу мясо - получается супер.

В скороварку надо класть меньше соли и специй.

На "Супе" можно готовить на пару (честно говоря - на любом режиме под давлением, но "Суп" всеми спокойнее воспринимается).

В режиме МВ крышку можно не закрывать( например для обжарки) чуть перекрыть и все.где требуется давление закрываете и закрываете клапан.

Для себя уяснила: на любом режиме Мультиварка ( овощи, запеканка, голубцы и т.п.) готовить с открытым клапаном и можно открывать крышку или вообще не закрывать, немного оставлять приоткрытой. Обжаривать можно на режиме овощи ( самый оптимальный). На режимах СВ все работает с закрытым клапаном под давлением. Все каши варю на режиме каша, все хлопья варю на режиме овощи, для супа мясо в режиме мясо или курица, далее обжарки овощей, далее все соединяю и режим суп.

Как уменьшить время приготовления: нужно довести до 59 минут, и отсчет с нуля начинается.


\chapter*{Режимы}

Филипс Бранд

Крупы 14мин Рис 14мин

Каша 13мин Каша/варка на пару 13мин

Курица 15мин Тушение 15мин

Мясо 30мин Мясо 30мин

Холодец 30мин Б обовые 30мин

Суп 20мин Суп 20мин

Овощи 20мин ---

Выпечка 45мин - я бы перенесла в раздел "без давления", как и есть у Бранда 



Режимы без давления


Филипс Бранд

Рагу 23мин Жарка: курица 23мин

Запеканка 18мин Жарка: мясо 18мин

Рыба 15мин Жарка: рыба 15мин

Овощи 3мин Жарка: овощи 15мин

Голубцы 20мин \/ Жарка: морепродукты 20мин /

Плов 20мин /\ Жарка: злаки 20мин / или наоборот?

-- Выпечка 45мин


Функция	Температурный режим	Время приготовления, установленное по умолчанию, мин	Настраивоемое время приготовления, мин	Давление 55+/-7 KПа Параметры острочки старта
Крупы	140-150℃	14	14-59	+	24 часа
Суп	140-150℃	20	20-59	+	24 часа
Холодец	140-150℃	30	25-59	+	24 часа
Курица	140-150℃	15	 12-59	+	24 часа
Каша	140-150℃	13	 1-59	+	24 часа
Овощи	140-150℃	30	 5-59	+	24 часа
Мясо	140-150℃	30	30-59	+	24 часа
Выпечка	120-130℃	45	40-59	+	нет

Рагу	120-130℃	23	20-59	нет	24 часа
Запеканка	150-160℃	18	18-59	нет	24 часа
Рыба	120-130℃	15	15-59	нет	24 часа
Овощи 120-130℃	3	 1-59	нет	24 часа
Голубцы	120-130℃	20	20-59	нет	24 часа

\chapter*{Рецепты}
\section*{Курица по-тайски (адаптация из книги рецептов)}

Сразу ставим режим Рагу на 20 мин. Пока идет нагрев, обжариваем курицу и кладем туда все ингредиенты, закрываем крышку. 

Создается давление, идет работа под давлением 20 мин. - получается классное блюдо. (10 мин. для вермишели все равно мало) 

\todo[inline]{Следует проверить}

\section*{Куриные бёдрышки в луковом соусе}

Подготовка: помыть, лук почистить-порезать-уложить-присолить, чуть зиры, чуть перчика, чуть сушеной паприки --- 5 минут. 

Закладка: режим скороварка:курица (стандартное время 15 минут) --- старт.

Через 15 минут пилик --- готово --- 5 минут для сброса давления - открываем...

Красиво, готово, пахнет вкусно. Все запахи там внутри сконцентрированы. 

Лучек, прада, не весь "разошелся в соус" дала еще 15 минут (многовато, 5 минут был бы достаточно).

\section*{Кролик}

Кролика разрезаем на порционные куски, достаточно большие, натираем немного солью и давленным чесноком (чеснока не жалеем - несколько зубчиков). Измельчаем штук пять-шесть луковиц. В кастрюлю насыпаем часть лука, кладём куски кролика, сверху снова луком, хорошенько утрамбовываем, прикрываем крышкой и ставим кастрюлю на сутки на холод, но не мороз --- на веранду или в холодильник. Через сутки можно кролика тушить, при этом добавляем в чашу топленое сало, а перед самым концом --- сливки. Кролик мягчайший получается. Может быть не обязательно сливками заливать, но дать ему полежать в луке сутки - и будет вам мягкий кролик.

Такой отлежавшийся кролик (половина кролика) кладется в чашу вместе с луком, пару ложек топленого масла, 1 мультистаканчик воды, на режим, соответствующий мясу минут на 40. Клапан закрыт. После сброса давления крышку открыть, добавить сливки в образовавшийся соус (можно и целую баночку или половину - здесь нет правил никаких), полить равномерно, кролик может костями поцарапать чашу, так что аккуратнее с ним, включить опять режим "мясо", крышку оставить открытой, дать соусу со сливками закипеть, сразу выключить св. Хорош с картофельным пюре.

\section*{Шарлотка}
Ингредиенты: 2яйца, 1 ст. сахара, 1 ст. муки, ч.л. соды, яблоки. 

Выпекать 45 минут на режиме выпечка.

\end{document}