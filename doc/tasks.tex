%!TEX root = thesis.tex

\newpage
\clearpage
\thispagestyle{empty}

\begin{center}
    Белорусский государственный университет

    \vspace{1em}

    Кафедра теории вероятностей и математической статистики

    \vspace{1em}

    \begin{minipage}[h]{0.8\textwidth}
    Утверждаю

    Заведующий кафедрой \rule{3.5cm}{0.4pt} Труш Н.Н.

    \vspace{1em}

    Дата
    \end{minipage}

    \vspace{2em}

    \textbf{ЗАДАНИЕ НА ДИПЛОМНУЮ РАБОТУ}
\end{center}

\vspace{1em}

Обучающемуся(студенту) \textit{Павлову А.С.}.
\begin{enumerate}
\item Тема дипломной работы

\textit{Анализ и прогнозирование гидрологических данных}

Утверждена приказом ректора БГУ от \rule{2cm}{0.4pt}№\rule{1cm}{0.4pt}

\item Исходные данные к дипломной работе
    \begin{enumerate}
        \item Cressie N. Statistics for Spatial Data. --- New York. --- Wiley, 1991. --- 900 p
        \item Савельев А.А. Геостатистический анализ данных в экологии и природопользовании (с применением пакета R) / Савельев А.А., Мухарамова С.С., Пилюгин А.Г., Чижикова Н.А. --- Казань: Казанский университет, 2012 --- 120 с.
        \item Robert H. Shumway, David S. Stoffer. Time series and Its Applications: With R Examples (Springer Texts in Statistics). Springer Science+Business Media, LLC 2011, 3d edition, 2011. – 576 р.
        \item Paul Teetor. R Cookbook (O’Reilly Cookbooks). O’Reilly Media, 1 edition, 2011. – 438 р.
        \item База данных характеристик водной системы озера Баторино (Нарочанская биологическая станция им. Г.Г.Винберга).
    \end{enumerate}

\item Перечень подлежащих разработке вопросов или краткое содержание расчетно-пояснительной записки:

    С использованием среды программирования R осуществить:
    \begin{enumerate}
        \item предварительный статистический анализ гидроэкологических данных озера Баторино.
        \item вариограммный анализ временного ряда: построение оценок семивариограммы,  подбор моделей семивариограммы.
        \newpage
        \clearpage
        \thispagestyle{empty}
        \item исследование статистических свойств оценки вариограммы гауссовского случайного процесса.
        \item прогнозирование значений временного ряда с помощью интерполяционного метода кригинг. Исследование точности прогноза в зависимости от оценки вариограммы и модели вариограммы, лежащих в основе метода кригинг.
    \end{enumerate}
\item Перечень графического материала (с точным указанием обязательных чертежей и графиков)
\item Консультанты по дипломной работе с указанием относящихся к ним разделов \textit{Цеховая Т.В.}
\item Примерный график выполнения дипломной работы
    \begin{itemize}
        \item 31 марта 2015 г. промежуточный отчет
        \item 28 апреля 2015 г. промежуточный отчет
        \item 12 мая 2015 г. доклад о проделанной работе
    \end{itemize}
\item Дата выдачи задания \rule{9.93cm}{0.4pt}
\item Срок сдачи законченной дипломной работы \rule{5cm}{0.4pt}
\end{enumerate}

\vspace{1em}

Руководитель \rule{3.5cm}{0.4pt} Т.В. Цеховая

\vspace{0.5em}

Подпись обучающегося \rule{4.5cm}{0.4pt}

\vspace{0.5em}

Дата