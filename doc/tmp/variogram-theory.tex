% (PARTIAL)
\documentclass[a4paper]{article}

\usepackage[russian]{babel}
\usepackage[utf8x]{inputenc}
\usepackage{amssymb,amsthm,amsmath,amscd}
\usepackage{graphicx}

%Определение полей
\setlength{\oddsidemargin}{0cm}% 1in=2.54см
\setlength{\hoffset}{0.46cm}% 1in+\hoffset=3cm = левое поле;

\setlength{\textwidth}{17cm}% 21cm-3cm(левое поле)-1cm(правое поле)=17cm;

\setlength{\headheight}{0cm}%
\setlength{\topmargin}{0cm}%
\setlength{\headsep}{0cm}%
\setlength{\voffset}{-0.54cm}% 1in+\voffset=2cm = верхнее поле;

\setlength{\textheight}{25.7cm}% 29.7cm-2cm(верхнее поле)-2cm(нижнее поле)=25.7cm;

\title{Многомерный статистический анализ}
\author{Хаткевич Людмила Анатольевна}
\begin{document}


\section*{Оценка вариограммы гауссовского случайного процесса} % (fold)
\label{sec:_variogram}

Пусть $ (\Omega, \mathcal{F}, P) $ --- вероятностное пространство.

\textit{Случайным процессом} $ x(t) = x(\omega, t) $ называется семейство случайных величин, заданных на вероятностном пространстве $ (\Omega, \mathcal{F}, P) $, где $ \omega \in \Omega, t \in T $, $ T $ --- некоторое параметрическое множество.

\textit{Математическим ожиданием} случайного процесса $ x(t), t \in T $ называется функция вида
\begin{equation}
	m(t) = Ex(t) = \int \limits_{\mathbb{R}} x \, dF(x;t), t \in T.
\end{equation}

\textit{Дисперсией} случайного процесса $ x(t), t \in T $ называется функция вида:
\begin{equation}
	V(t) = V( x(t) ) = E(x(t) - m(t))^2 = \int \limits_{\mathbb{R}} (x - m(t))^2 \, dF(x; t).
\end{equation}

\textit{Корреляционной функцией} случайного процесса $ x(t), t \in T $ называется функция вида:
\begin{equation}
	r(t_1, t_2) = E(x(t_1)x(t_2)) = \iint \limits_{\mathbb{R}^2} x(t_1)x(t_2) \, dF(x(t_1), x(t_2); t_1, t_2)
\end{equation}

\textit{Ковариационной функцией} случайного процесса $ x(t), t \in T $ называется функция вида:
\begin{equation}
	R(t_1, t_2) = E \{ (x(t_1) - m(t_1)) (x(t_2) - m(t_2)) \} = \iint \limits_{\mathbb{R}^2} (x(t_1) - m(t_1)) (x(t_2) - m(t_2)) \, dF(x(t_1), x(t_2); t_1, t_2)
\end{equation}

Случайный процесс $ X(s), s \in \mathbb{Z} = \{0, \pm 1, \pm 2, \dots \} $, называется внутренне стационарным, если справедливы следующие равенства:
\begin{equation}
	E \{ X(s_1) - X(s_2) \} = 0, \quad V \{ X(s_1) - X(s_2) \} = 2 \gamma(s_1 - s_2),
\end{equation}
где $2 \gamma(s_1 - s_2)$ --- вариограмма рассматриваемого процесса, $s_1,s_2 \in \mathbb{Z}$.

Пусть $ X(s), s \in \mathbb{Z} $ --- внутренне стационарный гауссовский случайный процесс с нулевым математическим ожиданием, дисперсией $ \sigma^2 $ и неизвестной вариограммой.
\begin{equation}
	2 \gamma(h) = V \{ X(s+h) - X(s) \}, ~ s,h \in \mathbb{Z}.
\end{equation}

В качестве оценки вариограммы рассмотрим статистику вида:
\begin{equation}
	2 \tilde{\gamma}(h) = \frac{1}{n-h} \sum_{s=1}^{n-h}(X(s+h) - X(s))^2, \quad h = \overline{0, n-1}.
\end{equation}

Вычислим математическое ожидание введённой оценки %TODO: WRITE THIS LATER%

Таким образом оценка является несмещённой.

Далее, найдём второй момент:


\begin{eqnarray}
	& cov(2 \tilde{\gamma}(h_1), 2 \tilde{\gamma}(h_2)) = E((2 \tilde{\gamma}(h_1) - E(2 \tilde{\gamma}(h_1))) (2 \tilde{\gamma}(h_2) - E(2 \tilde{\gamma}(h_2)))) = \\
	& = E(\frac{1}{n-h_1} \sum_{s=1}^{n-h_1}((x(s+h_1) - x(s))^2 - E((x(s+h_1) - x(s))^2)) \times \\
	& \times \frac{1}{n-h_2} \sum_{t=1}^{n-h_2}((x(t+h_2) - x(t))^2 - E((x(t+h_2) - x(t))^2))) = \\
	& = \frac{1}{(n-h_1)(n-h_2)}\sum_{s=1}^{n-h_1}\sum_{t=1}^{n-h_2} cov((x(s+h_1) - x(s))^2, (x(t+h_2) - x(t))^2) = \\
	& = [\text{по определению} ~ cov(a,b) = corr(a,b)\sqrt{V(a)V(b)}] = \\
	& = \frac{1}{(n-h_1)(n-h_2)}\sum_{s=1}^{n-h_1}\sum_{t=1}^{n-h_2} corr((x(s+h_1) - x(s))^2, (x(t+h_2) - x(t))^2) \times \\
	& \times \sqrt{(V((x(s+h_1) - x(s))^2) V((x(t+h_2) - x(t))^2))} = \\
	& = [\text{используем равенство (a) со страницы 2, TODO: добавить ссылку на источник}] = \\
	& = \frac{2 (2\gamma(h_1))(2\gamma(h_2))}{(n-h_1)(n-h_2)}\sum_{s=1}^{n-h_1}\sum_{t=1}^{n-h_2} corr((x(s+h_1) - x(s))^2, (x(t+h_2) - x(t))^2)
\end{eqnarray}

\end{document}

% section _ (end)