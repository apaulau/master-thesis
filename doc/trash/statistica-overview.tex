% (PARTIAL)
\newpage

\chapter{Обзор возможностей статистической обработки данных в пакете \textbf{STATISTICA}}

\todo[inline, color=red!60]{Решить, что делать с данным пунктом: удалить вовсе, как-либо изменить, или оставить все как есть.}
\textbf{STATISTICA} — это универсальная интегрированная система, предназначенная для статистического анализа и визуализации данных, управления базами данных и разработки пользовательских приложений, содержащая широкий набор процедур анализа для применения в научных исследованиях, технике, бизнесе, а также специальные методы добычи данных.

В системе \textbf{STATISTICA} реализовано множество мощных языков программирования, которые снабжены специальными средствами поддержки. С их помощью легко создаются законченные пользовательские решения и встраиваются в различные другие приложения или вычислительные среды. 
 
В пакете представлены несколько сотен типов графиков 2D, 3D и 4D, матрицы и пиктограммы; предоставляется возможность разработки собственного дизайна графика. Средства управления графиками позволяют работать одновременно с несколькими графиками, изменять размеры сложных объектов, добавлять художественную перспективу и ряд специальных эффектов, разбивку страниц и быструю перерисовку. Например, 3D-графики можно вращать, накладывать друг на друга, сжимать или увеличивать.

\textbf{STATISTICA} обладает огромными возможностями для построения графиков непосредственно из таблиц исходных данных и таблиц результатов. Построение графических объектов и анализ данных в пакете тесно интегрированы. После получения результатов статистического анализа их можно с лёгкостью представить графически посредством команды Быстрые статистические графики. В разных модулях системы имеются свои специальные графики, учитывающие особенности получаемых в них результатов.

Данный пакет позволяет проводить исчерпывающий, всесторонний анализ данных для научного, коммерческого и инженерного применения. Программа обладает превосходными средствами представления результатов анализа в виде таблиц и графиков, позволяет автоматически создавать отчёты по проделанной работе. Система подсказок составлена настолько продуманно и так удобна в обращении, что с её помощью можно обучаться не только работе с самим пакетом, но и современным методам статистического анализа.